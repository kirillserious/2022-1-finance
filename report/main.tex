\documentclass[a4paper, 12pt]{article}
\usepackage{a4wide}
% Необходимые пакеты для компиляции русского языка, картинок и прочего

\usepackage[utf8]{inputenc}                % Кодировка
\usepackage[main=russian, english]{babel}  % Русский язык
\usepackage[pdftex]{graphicx}              % Картинки
\usepackage{indentfirst}                   % Отступ перед абзацами

\usepackage{amsmath}  % Математические 
\usepackage{amssymb}  % формулы

\usepackage{tikz}                % Векторная графика
                                 %
\usepackage{pgfplots}            % % Нужно для вставки графиков из 
\pgfplotsset{compat=newest}      % % matlab2tikz
\usepgfplotslibrary{groupplots}
\usepgfplotslibrary{dateplot}
\usetikzlibrary{plotmarks}       % %
\usetikzlibrary{arrows.meta}     % %
\usepgfplotslibrary{patchplots}  % %
\usepackage{grffile}             % %

\usepackage{caption} % Чтобы можно было вставлять формулы к подписям рисунков

\usepackage[unicode]{hyperref}                                         % Ссылки и русские закладки
\hypersetup {                                                          %
    pdftitle={Отчёт по практикуму},                                    % Название документа
    pdfsubject={Динамическое программирование и процессы управления},  % Тема документа
    pdfauthor={Егоров Кирилл Юлианович},                 % Автор документа
    pdfcreator={Кафедра системного анализа ВМК МГУ},     % Создатель документа
    pdfproducer={LaTeX},                                 % Программа, создавшая документ
    hidelinks                                            % Скрывает рамку вокруг ссылок
}


\usepackage{nicefrac}
\usepackage{amsthm}  % Красивый внешний вид теорем, определений и доказательств
\newtheoremstyle{def}
        {\topsep}
        {\topsep}
        {\normalfont}
        {\parindent}
        {\bfseries}
        {.}
        {.5em}
        {}
\theoremstyle{def}
\newtheorem{definition}{Определение}
\newtheorem{example}{Пример}

\newtheoremstyle{th}
        {\topsep}
        {\topsep}
        {\itshape}
        {\parindent}
        {\bfseries}
        {.}
        {.5em}
        {}
\theoremstyle{th}
\newtheorem{theorem}{Теорема}
\newtheorem{lemma}{Лемма}
\newtheorem{assertion}{Утверждение}

\newtheoremstyle{rem}
        {0.5\topsep}
        {0.5\topsep}
        {\normalfont}
        {\parindent}
        {\itshape}
        {.}
        {.5em}
        {}
\theoremstyle{rem}
\newtheorem{remark}{Замечание}

% Новое доказательство
\renewenvironment{proof}{\parД о к а з а т е л ь с т в о.}{\hfill$\blacksquare$}

\begin{document}
    % This file was created with tikzplotlib v0.10.1.
\begin{tikzpicture}

\definecolor{darkgray176}{RGB}{176,176,176}
\definecolor{darkkhaki174189103}{RGB}{174,189,103}
\definecolor{darkkhaki18218999}{RGB}{182,189,99}
\definecolor{darkkhaki20618790}{RGB}{206,187,90}
\definecolor{darkkhaki21418686}{RGB}{214,186,86}
\definecolor{darkseagreen107190130}{RGB}{107,190,130}
\definecolor{darkseagreen138190117}{RGB}{138,190,117}
\definecolor{darkseagreen148190113}{RGB}{148,190,113}
\definecolor{darkslateblue5342134}{RGB}{53,42,134}
\definecolor{darkslateblue5357165}{RGB}{53,57,165}
\definecolor{darkslateblue5454159}{RGB}{54,54,159}
\definecolor{dodgerblue14119219}{RGB}{14,119,219}
\definecolor{dodgerblue14144209}{RGB}{14,144,209}
\definecolor{dodgerblue16122217}{RGB}{16,122,217}
\definecolor{dodgerblue16142210}{RGB}{16,142,210}
\definecolor{dodgerblue19129213}{RGB}{19,129,213}
\definecolor{dodgerblue20132211}{RGB}{20,132,211}
\definecolor{dodgerblue3108224}{RGB}{3,108,224}
\definecolor{dodgerblue498224}{RGB}{4,98,224}
\definecolor{dodgerblue6111222}{RGB}{6,111,222}
\definecolor{gold24522332}{RGB}{245,223,32}
\definecolor{gold24621836}{RGB}{246,218,36}
\definecolor{gold25320248}{RGB}{253,202,48}
\definecolor{goldenrod25419753}{RGB}{254,197,53}
\definecolor{lightseagreen13173186}{RGB}{13,173,186}
\definecolor{lightseagreen24177177}{RGB}{24,177,177}
\definecolor{lightseagreen56185157}{RGB}{56,185,157}
\definecolor{lightseagreen6156207}{RGB}{6,156,207}
\definecolor{lightseagreen6160204}{RGB}{6,160,204}
\definecolor{mediumaquamarine98190134}{RGB}{98,190,134}
\definecolor{mediumseagreen67186151}{RGB}{67,186,151}
\definecolor{royalblue1392221}{RGB}{13,92,221}
\definecolor{royalblue4175201}{RGB}{41,75,201}
\definecolor{royalblue4770191}{RGB}{47,70,191}
\definecolor{sandybrown23618576}{RGB}{236,185,76}
\definecolor{sandybrown24318571}{RGB}{243,185,71}
\definecolor{yellow24724417}{RGB}{247,244,17}
\definecolor{yellow24825013}{RGB}{248,250,13}

\begin{axis}[
hide x axis,
hide y axis,
tick align=outside,
tick pos=left,
x grid style={darkgray176},
xmin=89, xmax=111,
xtick style={color=black},
y grid style={darkgray176},
ymin=89, ymax=111,
ytick style={color=black},
yticklabel style={anchor=center}
]
\path [fill=yellow24724417]
(axis cs:-0.0209432215847148,0.0634238103078434)
--(axis cs:-0.0164203754046307,0.0622493061972241)
--(axis cs:-0.0137604888890911,0.0686444311803462)
--(axis cs:-0.0182825190376708,0.0698025911723151)
--cycle;
\path [fill=gold24621836]
(axis cs:-0.0236120337857995,0.0570255864873861)
--(axis cs:-0.0190883876450915,0.0558346438268063)
--(axis cs:-0.0164203754046307,0.0622493061972241)
--(axis cs:-0.0209432215847148,0.0634238103078434)
--cycle;
\path [fill=yellow24724417]
(axis cs:-0.0164203754046307,0.0622493061972241)
--(axis cs:-0.0118768092589539,0.0610694214736476)
--(axis cs:-0.00921777413226355,0.067480973548836)
--(axis cs:-0.0137604888890911,0.0686444311803462)
--cycle;
\path [fill=goldenrod25419753]
(axis cs:-0.0262889927908485,0.0506078320718691)
--(axis cs:-0.0217645629158052,0.0494003557948235)
--(axis cs:-0.0190883876450915,0.0558346438268063)
--(axis cs:-0.0236120337857995,0.0570255864873861)
--cycle;
\path [fill=gold24621836]
(axis cs:-0.0190883876450915,0.0558346438268063)
--(axis cs:-0.0145439861469301,0.0546382368925282)
--(axis cs:-0.0118768092589539,0.0610694214736476)
--(axis cs:-0.0164203754046307,0.0622493061972241)
--cycle;
\path [fill=yellow24724417]
(axis cs:-0.0118768092589539,0.0610694214736476)
--(axis cs:-0.00731238043800453,0.0598841190779039)
--(axis cs:-0.00465423251897662,0.0663121818458762)
--(axis cs:-0.00921777413226355,0.067480973548836)
--cycle;
\path [fill=sandybrown23618576]
(axis cs:-0.0289741359577366,0.044170457180381)
--(axis cs:-0.0244489387357876,0.0429463515772432)
--(axis cs:-0.0217645629158052,0.0494003557948235)
--(axis cs:-0.0262889927908485,0.0506078320718691)
--cycle;
\path [fill=goldenrod25419753]
(axis cs:-0.0217645629158052,0.0494003557948235)
--(axis cs:-0.0172193422574711,0.0481873308899292)
--(axis cs:-0.0145439861469301,0.0546382368925282)
--(axis cs:-0.0190883876450915,0.0558346438268063)
--cycle;
\path [fill=gold24621836]
(axis cs:-0.0145439861469301,0.0546382368925282)
--(axis cs:-0.00997868611851274,0.0534363279913741)
--(axis cs:-0.00731238043800453,0.0598841190779039)
--(axis cs:-0.0118768092589539,0.0610694214736476)
--cycle;
\path [fill=darkkhaki20618790]
(axis cs:-0.0316723359196522,0.0379835820193918)
--(axis cs:-0.0271457064015051,0.0367431822567889)
--(axis cs:-0.0244489387357876,0.0429463515772432)
--(axis cs:-0.0289741359577366,0.044170457180381)
--cycle;
\path [fill=yellow24724417]
(axis cs:-0.00731238043800453,0.0598841190779039)
--(axis cs:-0.0027269449185277,0.0586933616096707)
--(axis cs:-6.97204936989334e-05,0.0651380193047343)
--(axis cs:-0.00465423251897662,0.0663121818458762)
--cycle;
\path [fill=sandybrown23618576]
(axis cs:-0.0244489387357876,0.0429463515772432)
--(axis cs:-0.0199029152713224,0.0417166122925429)
--(axis cs:-0.0172193422574711,0.0481873308899292)
--(axis cs:-0.0217645629158052,0.0494003557948235)
--cycle;
\path [fill=goldenrod25419753]
(axis cs:-0.0172193422574711,0.0481873308899292)
--(axis cs:-0.0126531871782629,0.046968719023301)
--(axis cs:-0.00997868611851274,0.0534363279913741)
--(axis cs:-0.0145439861469301,0.0546382368925282)
--cycle;
\path [fill=darkkhaki174189103]
(axis cs:-0.0343796381350612,0.0317758360888891)
--(axis cs:-0.0298515924450427,0.0305190389940308)
--(axis cs:-0.0271457064015051,0.0367431822567889)
--(axis cs:-0.0316723359196522,0.0379835820193918)
--cycle;
\path [fill=gold24621836]
(axis cs:-0.00997868611851274,0.0534363279913741)
--(axis cs:-0.00539234306717235,0.0522288790826846)
--(axis cs:-0.0027269449185277,0.0586933616096707)
--(axis cs:-0.00731238043800453,0.0598841190779039)
--cycle;
\path [fill=darkkhaki20618790]
(axis cs:-0.0271457064015051,0.0367431822567889)
--(axis cs:-0.0225982088071792,0.0354970641652098)
--(axis cs:-0.0199029152713224,0.0417166122925429)
--(axis cs:-0.0244489387357876,0.0429463515772432)
--cycle;
\path [fill=yellow24724417]
(axis cs:-0.0027269449185277,0.0586933616096707)
--(axis cs:0.00187964265145435,0.0574971113235797)
--(axis cs:0.00453590682147324,0.0639584488199982)
--(axis cs:-6.97204936989334e-05,0.0651380193047343)
--cycle;
\path [fill=sandybrown23618576]
(axis cs:-0.0199029152713224,0.0417166122925429)
--(axis cs:-0.0153359214603152,0.0404812003448751)
--(axis cs:-0.0126531871782629,0.046968719023301)
--(axis cs:-0.0172193422574711,0.0481873308899292)
--cycle;
\path [fill=darkseagreen138190117]
(axis cs:-0.0370960887340661,0.0255471132003501)
--(axis cs:-0.0325666431858908,0.0242738148315304)
--(axis cs:-0.0298515924450427,0.0305190389940308)
--(axis cs:-0.0343796381350612,0.0317758360888891)
--cycle;
\path [fill=goldenrod25419753]
(axis cs:-0.0126531871782629,0.046968719023301)
--(axis cs:-0.00806595271440819,0.0457444815071217)
--(axis cs:-0.00539234306717235,0.0522288790826846)
--(axis cs:-0.00997868611851274,0.0534363279913741)
--cycle;
\path [fill=darkkhaki174189103]
(axis cs:-0.0298515924450427,0.0305190389940308)
--(axis cs:-0.0253026367778631,0.0292564381594692)
--(axis cs:-0.0225982088071792,0.0354970641652098)
--(axis cs:-0.0271457064015051,0.0367431822567889)
--cycle;
\path [fill=gold24621836]
(axis cs:-0.00539234306717235,0.0522288790826846)
--(axis cs:-0.000784811165133218,0.0510158517743058)
--(axis cs:0.00187964265145435,0.0574971113235797)
--(axis cs:-0.0027269449185277,0.0586933616096707)
--cycle;
\path [fill=darkkhaki20618790]
(axis cs:-0.0225982088071792,0.0354970641652098)
--(axis cs:-0.0180296984983596,0.0342451881104563)
--(axis cs:-0.0153359214603152,0.0404812003448751)
--(axis cs:-0.0199029152713224,0.0417166122925429)
--cycle;
\path [fill=mediumaquamarine98190134]
(axis cs:-0.0398217341659482,0.0192973068629177)
--(axis cs:-0.0352909052636774,0.0180074025033099)
--(axis cs:-0.0325666431858908,0.0242738148315304)
--(axis cs:-0.0370960887340661,0.0255471132003501)
--cycle;
\path [fill=yellow24724417]
(axis cs:0.00187964265145435,0.0574971113235797)
--(axis cs:0.00650752896800126,0.0562953301252277)
--(axis cs:0.0091627956420781,0.0627734329436676)
--(axis cs:0.00453590682147324,0.0639584488199982)
--cycle;
\path [fill=sandybrown23618576]
(axis cs:-0.0153359214603152,0.0404812003448751)
--(axis cs:-0.0107478118661916,0.0392400763923686)
--(axis cs:-0.00806595271440819,0.0457444815071217)
--(axis cs:-0.0126531871782629,0.046968719023301)
--cycle;
\path [fill=darkseagreen138190117]
(axis cs:-0.0325666431858908,0.0242738148315304)
--(axis cs:-0.0280162456927906,0.0229946265415244)
--(axis cs:-0.0253026367778631,0.0292564381594692)
--(axis cs:-0.0298515924450427,0.0305190389940308)
--cycle;
\path [fill=goldenrod25419753]
(axis cs:-0.00806595271440819,0.0457444815071217)
--(axis cs:-0.00345749256060503,0.0445145792955478)
--(axis cs:-0.000784811165133218,0.0510158517743058)
--(axis cs:-0.00539234306717235,0.0522288790826846)
--cycle;
\path [fill=darkkhaki174189103]
(axis cs:-0.0253026367778631,0.0292564381594692)
--(axis cs:-0.0207326259586362,0.0279879932906959)
--(axis cs:-0.0180296984983596,0.0342451881104563)
--(axis cs:-0.0225982088071792,0.0354970641652098)
--cycle;
\path [fill=lightseagreen56185157]
(axis cs:-0.0425566211954085,0.0130263098774718)
--(axis cs:-0.0380244256354378,0.0117196940282094)
--(axis cs:-0.0352909052636774,0.0180074025033099)
--(axis cs:-0.0398217341659482,0.0192973068629177)
--cycle;
\path [fill=gold24621836]
(axis cs:-0.000784811165133218,0.0510158517743058)
--(axis cs:0.00384405676594559,0.0497972073185198)
--(axis cs:0.00650752896800126,0.0562953301252277)
--(axis cs:0.00187964265145435,0.0574971113235797)
--cycle;
\path [fill=darkkhaki20618790]
(axis cs:-0.0180296984983596,0.0342451881104563)
--(axis cs:-0.013440029496969,0.0329875140912047)
--(axis cs:-0.0107478118661916,0.0392400763923686)
--(axis cs:-0.0153359214603152,0.0404812003448751)
--cycle;
\path [fill=lightseagreen13173186]
(axis cs:-0.0451964078217572,0.00272723814681047)
--(axis cs:-0.0406730920118061,0.00139758976091727)
--(axis cs:-0.0380244256354378,0.0117196940282094)
--(axis cs:-0.0425566211954085,0.0130263098774718)
--cycle;
\path [fill=mediumaquamarine98190134]
(axis cs:-0.0352909052636774,0.0180074025033099)
--(axis cs:-0.0307390823824777,0.0167115212627539)
--(axis cs:-0.0280162456927906,0.0229946265415244)
--(axis cs:-0.0325666431858908,0.0242738148315304)
--cycle;
\path [fill=yellow24724417]
(axis cs:0.00650752896800126,0.0562953301252277)
--(axis cs:0.0111568620868264,0.0550879795671337)
--(axis cs:0.0138110935367615,0.0615829338811909)
--(axis cs:0.0091627956420781,0.0627734329436676)
--cycle;
\path [fill=sandybrown23618576]
(axis cs:-0.0107478118661916,0.0392400763923686)
--(axis cs:-0.00613843970439106,0.0379932007285103)
--(axis cs:-0.00345749256060503,0.0445145792955478)
--(axis cs:-0.00806595271440819,0.0457444815071217)
--cycle;
\path [fill=lightseagreen6156207]
(axis cs:-0.0479433866074152,-0.00357869134563717)
--(axis cs:-0.0434187434603486,-0.00492521798947779)
--(axis cs:-0.0406730920118061,0.00139758976091727)
--(axis cs:-0.0451964078217572,0.00272723814681047)
--cycle;
\path [fill=darkseagreen138190117]
(axis cs:-0.0280162456927906,0.0229946265415244)
--(axis cs:-0.0234447505410178,0.0217095073679062)
--(axis cs:-0.0207326259586362,0.0279879932906959)
--(axis cs:-0.0253026367778631,0.0292564381594692)
--cycle;
\path [fill=goldenrod25419753]
(axis cs:-0.00345749256060503,0.0445145792955478)
--(axis cs:0.00117234094553327,0.0432789729805579)
--(axis cs:0.00384405676594559,0.0497972073185198)
--(axis cs:-0.000784811165133218,0.0510158517743058)
--cycle;
\path [fill=darkkhaki174189103]
(axis cs:-0.0207326259586362,0.0279879932906959)
--(axis cs:-0.016141413465454,0.0267136637193252)
--(axis cs:-0.013440029496969,0.0329875140912047)
--(axis cs:-0.0180296984983596,0.0342451881104563)
--cycle;
\path [fill=lightseagreen56185157]
(axis cs:-0.0380244256354378,0.0117196940282094)
--(axis cs:-0.0334711939968474,0.0104070135523512)
--(axis cs:-0.0307390823824777,0.0167115212627539)
--(axis cs:-0.0352909052636774,0.0180074025033099)
--cycle;
\path [fill=gold24621836]
(axis cs:0.00384405676594559,0.0497972073185198)
--(axis cs:0.0084944092706414,0.0485729066079188)
--(axis cs:0.0111568620868264,0.0550879795671337)
--(axis cs:0.00650752896800126,0.0562953301252277)
--cycle;
\path [fill=darkkhaki20618790]
(axis cs:-0.013440029496969,0.0329875140912047)
--(axis cs:-0.00882905446962072,0.0317240017347458)
--(axis cs:-0.00613843970439106,0.0379932007285103)
--(axis cs:-0.0107478118661916,0.0392400763923686)
--cycle;
\path [fill=lightseagreen13173186]
(axis cs:-0.0406730920118061,0.00139758976091727)
--(axis cs:-0.0361287942064315,6.17736260718122e-05)
--(axis cs:-0.0334711939968474,0.0104070135523512)
--(axis cs:-0.0380244256354378,0.0117196940282094)
--cycle;
\path [fill=mediumaquamarine98190134]
(axis cs:-0.0307390823824777,0.0167115212627539)
--(axis cs:-0.0261661192674414,0.0154096215032079)
--(axis cs:-0.0234447505410178,0.0217095073679062)
--(axis cs:-0.0280162456927906,0.0229946265415244)
--cycle;
\path [fill=dodgerblue16142210]
(axis cs:-0.0507309852001879,-0.00884355946596816)
--(axis cs:-0.0462023057106435,-0.0102054463828454)
--(axis cs:-0.0434187434603486,-0.00492521798947779)
--(axis cs:-0.0479433866074152,-0.00357869134563717)
--cycle;
\path [fill=yellow24724417]
(axis cs:0.0111568620868264,0.0550879795671337)
--(axis cs:0.0158277914390856,0.0538750208446377)
--(axis cs:0.0184809494429652,0.0603869134874473)
--(axis cs:0.0138110935367615,0.0615829338811909)
--cycle;
\path [fill=sandybrown23618576]
(axis cs:-0.00613843970439106,0.0379932007285103)
--(axis cs:-0.00150765682671258,0.0367405332779105)
--(axis cs:0.00117234094553327,0.0432789729805579)
--(axis cs:-0.00345749256060503,0.0445145792955478)
--cycle;
\path [fill=lightseagreen6156207]
(axis cs:-0.0434187434603486,-0.00492521798947779)
--(axis cs:-0.0388730763753739,-0.00627800132398585)
--(axis cs:-0.0361287942064315,6.17736260718122e-05)
--(axis cs:-0.0406730920118061,0.00139758976091727)
--cycle;
\path [fill=darkseagreen138190117]
(axis cs:-0.0234447505410178,0.0217095073679062)
--(axis cs:-0.0188520106624974,0.0204184159675274)
--(axis cs:-0.016141413465454,0.0267136637193252)
--(axis cs:-0.0207326259586362,0.0279879932906959)
--cycle;
\path [fill=goldenrod25419753]
(axis cs:0.00117234094553327,0.0432789729805579)
--(axis cs:0.00582369683924924,0.042037622787744)
--(axis cs:0.0084944092706414,0.0485729066079188)
--(axis cs:0.00384405676594559,0.0497972073185198)
--cycle;
\path [fill=darkkhaki174189103]
(axis cs:-0.016141413465454,0.0267136637193252)
--(axis cs:-0.0115288514137286,0.0254334083987482)
--(axis cs:-0.00882905446962072,0.0317240017347458)
--(axis cs:-0.013440029496969,0.0329875140912047)
--cycle;
\path [fill=lightseagreen56185157]
(axis cs:-0.0334711939968474,0.0104070135523512)
--(axis cs:-0.0288967794812641,0.0090882261284486)
--(axis cs:-0.0261661192674414,0.0154096215032079)
--(axis cs:-0.0307390823824777,0.0167115212627539)
--cycle;
\path [fill=gold24621836]
(axis cs:0.0084944092706414,0.0485729066079188)
--(axis cs:0.0131663962756578,0.0473429101712218)
--(axis cs:0.0158277914390856,0.0538750208446377)
--(axis cs:0.0111568620868264,0.0550879795671337)
--cycle;
\path [fill=darkkhaki20618790]
(axis cs:-0.00882905446962072,0.0317240017347458)
--(axis cs:-0.00419662471185163,0.0304546102926638)
--(axis cs:-0.00150765682671258,0.0367405332779105)
--(axis cs:-0.00613843970439106,0.0379932007285103)
--cycle;
\path [fill=lightseagreen13173186]
(axis cs:-0.0361287942064315,6.17736260718122e-05)
--(axis cs:-0.0315633680746269,-0.00128025327236272)
--(axis cs:-0.0288967794812641,0.0090882261284486)
--(axis cs:-0.0334711939968474,0.0104070135523512)
--cycle;
\path [fill=mediumaquamarine98190134]
(axis cs:-0.0261661192674414,0.0154096215032079)
--(axis cs:-0.0215718683019822,0.0141016611989669)
--(axis cs:-0.0188520106624974,0.0204184159675274)
--(axis cs:-0.0234447505410178,0.0217095073679062)
--cycle;
\path [fill=dodgerblue16142210]
(axis cs:-0.0462023057106435,-0.0102054463828454)
--(axis cs:-0.0416525345231005,-0.0115736760995666)
--(axis cs:-0.0388730763753739,-0.00627800132398585)
--(axis cs:-0.0434187434603486,-0.00492521798947779)
--cycle;
\path [fill=dodgerblue19129213]
(axis cs:-0.0535315373732499,-0.0141328926324829)
--(axis cs:-0.0489988329058464,-0.0155102683369765)
--(axis cs:-0.0462023057106435,-0.0102054463828454)
--(axis cs:-0.0507309852001879,-0.00884355946596816)
--cycle;
\path [fill=yellow24724417]
(axis cs:0.0158277914390856,0.0538750208446377)
--(axis cs:0.0205204678473873,0.0526564147917443)
--(axis cs:0.0231725136828355,0.0591853332626723)
--(axis cs:0.0184809494429652,0.0603869134874473)
--cycle;
\path [fill=sandybrown23618576]
(axis cs:-0.00150765682671258,0.0367405332779105)
--(axis cs:0.00314468629455727,0.0354820335920088)
--(axis cs:0.00582369683924924,0.042037622787744)
--(axis cs:0.00117234094553327,0.0432789729805579)
--cycle;
\path [fill=lightseagreen6156207]
(axis cs:-0.0388730763753739,-0.00627800132398585)
--(axis cs:-0.0343062384783924,-0.00763708505866078)
--(axis cs:-0.0315633680746269,-0.00128025327236272)
--(axis cs:-0.0361287942064315,6.17736260718122e-05)
--cycle;
\path [fill=darkseagreen138190117]
(axis cs:-0.0188520106624974,0.0204184159675274)
--(axis cs:-0.014237877619057,0.0191213106120835)
--(axis cs:-0.0115288514137286,0.0254334083987482)
--(axis cs:-0.016141413465454,0.0267136637193252)
--cycle;
\path [fill=goldenrod25419753]
(axis cs:0.00582369683924924,0.042037622787744)
--(axis cs:0.010496725544633,0.0407904885720435)
--(axis cs:0.0131663962756578,0.0473429101712218)
--(axis cs:0.0084944092706414,0.0485729066079188)
--cycle;
\path [fill=darkkhaki174189103]
(axis cs:-0.0115288514137286,0.0254334083987482)
--(axis cs:-0.00689479054031263,0.0241471858997253)
--(axis cs:-0.00419662471185163,0.0304546102926638)
--(axis cs:-0.00882905446962072,0.0317240017347458)
--cycle;
\path [fill=lightseagreen56185157]
(axis cs:-0.0288967794812641,0.0090882261284486)
--(axis cs:-0.0243010339212373,0.0077632890403529)
--(axis cs:-0.0215718683019822,0.0141016611989669)
--(axis cs:-0.0261661192674414,0.0154096215032079)
--cycle;
\path [fill=gold24621836]
(axis cs:0.0131663962756578,0.0473429101712218)
--(axis cs:0.0178601691059375,0.0461071781690331)
--(axis cs:0.0205204678473873,0.0526564147917443)
--(axis cs:0.0158277914390856,0.0538750208446377)
--cycle;
\path [fill=darkkhaki20618790]
(axis cs:-0.00419662471185163,0.0304546102926638)
--(axis cs:0.000457409867863144,0.0291792986364568)
--(axis cs:0.00314468629455727,0.0354820335920088)
--(axis cs:-0.00150765682671258,0.0367405332779105)
--cycle;
\path [fill=lightseagreen13173186]
(axis cs:-0.0315633680746269,-0.00128025327236272)
--(axis cs:-0.0269766659215075,-0.00262853434994103)
--(axis cs:-0.0243010339212373,0.0077632890403529)
--(axis cs:-0.0288967794812641,0.0090882261284486)
--cycle;
\path [fill=mediumaquamarine98190134]
(axis cs:-0.0215718683019822,0.0141016611989669)
--(axis cs:-0.0169561804919516,0.0127875979321411)
--(axis cs:-0.014237877619057,0.0191213106120835)
--(axis cs:-0.0188520106624974,0.0204184159675274)
--cycle;
\path [fill=dodgerblue16142210]
(axis cs:-0.0416525345231005,-0.0115736760995666)
--(axis cs:-0.0370815239461359,-0.0129482930306274)
--(axis cs:-0.0343062384783924,-0.00763708505866078)
--(axis cs:-0.0388730763753739,-0.00627800132398585)
--cycle;
\path [fill=dodgerblue19129213]
(axis cs:-0.0489988329058464,-0.0155102683369765)
--(axis cs:-0.0444449687170643,-0.0168940739524316)
--(axis cs:-0.0416525345231005,-0.0115736760995666)
--(axis cs:-0.0462023057106435,-0.0102054463828454)
--cycle;
\path [fill=dodgerblue14119219]
(axis cs:-0.0563451336256558,-0.0194468618225809)
--(axis cs:-0.0518084158359676,-0.020839856128277)
--(axis cs:-0.0489988329058464,-0.0155102683369765)
--(axis cs:-0.0535315373732499,-0.0141328926324829)
--cycle;
\path [fill=yellow24724417]
(axis cs:0.0205204678473873,0.0526564147917443)
--(axis cs:0.0252350435420258,0.0514321218769064)
--(axis cs:0.0278859379793517,0.0579781543483271)
--(axis cs:0.0231725136828355,0.0591853332626723)
--cycle;
\path [fill=sandybrown23618576]
(axis cs:0.00314468629455727,0.0354820335920088)
--(axis cs:0.00781874058273858,0.0342176608447215)
--(axis cs:0.010496725544633,0.0407904885720435)
--(axis cs:0.00582369683924924,0.042037622787744)
--cycle;
\path [fill=lightseagreen6156207]
(axis cs:-0.0343062384783924,-0.00763708505866078)
--(axis cs:-0.0297180815240209,-0.00900251331109419)
--(axis cs:-0.0269766659215075,-0.00262853434994103)
--(axis cs:-0.0315633680746269,-0.00128025327236272)
--cycle;
\path [fill=darkseagreen138190117]
(axis cs:-0.014237877619057,0.0191213106120835)
--(axis cs:-0.00960220158643413,0.0178181491836184)
--(axis cs:-0.00689479054031263,0.0241471858997253)
--(axis cs:-0.0115288514137286,0.0254334083987482)
--cycle;
\path [fill=goldenrod25419753]
(axis cs:0.010496725544633,0.0407904885720435)
--(axis cs:0.0151915788908389,0.0395375298134116)
--(axis cs:0.0178601691059375,0.0461071781690331)
--(axis cs:0.0131663962756578,0.0473429101712218)
--cycle;
\path [fill=darkkhaki174189103]
(axis cs:-0.00689479054031263,0.0241471858997253)
--(axis cs:-0.00223908018739899,0.0228549544059173)
--(axis cs:0.000457409867863144,0.0291792986364568)
--(axis cs:-0.00419662471185163,0.0304546102926638)
--cycle;
\path [fill=lightseagreen56185157]
(axis cs:-0.0243010339212373,0.0077632890403529)
--(axis cs:-0.0196838077642421,0.00643215917260348)
--(axis cs:-0.0169561804919516,0.0127875979321411)
--(axis cs:-0.0215718683019822,0.0141016611989669)
--cycle;
\path [fill=gold24621836]
(axis cs:0.0178601691059375,0.0461071781690331)
--(axis cs:0.0225758805010002,0.04486567038954)
--(axis cs:0.0252350435420258,0.0514321218769064)
--(axis cs:0.0205204678473873,0.0526564147917443)
--cycle;
\path [fill=darkkhaki20618790]
(axis cs:0.000457409867863144,0.0291792986364568)
--(axis cs:0.00513320076432047,0.0278980252530939)
--(axis cs:0.00781874058273858,0.0342176608447215)
--(axis cs:0.00314468629455727,0.0354820335920088)
--cycle;
\path [fill=lightseagreen13173186]
(axis cs:-0.0269766659215075,-0.00262853434994103)
--(axis cs:-0.0223685386723843,-0.00398311342781738)
--(axis cs:-0.0196838077642421,0.00643215917260348)
--(axis cs:-0.0243010339212373,0.0077632890403529)
--cycle;
\path [fill=mediumaquamarine98190134]
(axis cs:-0.0169561804919516,0.0127875979321411)
--(axis cs:-0.0123189054495317,0.0114673888880697)
--(axis cs:-0.00960220158643413,0.0178181491836184)
--(axis cs:-0.014237877619057,0.0191213106120835)
--cycle;
\path [fill=dodgerblue16142210]
(axis cs:-0.0370815239461359,-0.0129482930306274)
--(axis cs:-0.0324891249061847,-0.0143293420061684)
--(axis cs:-0.0297180815240209,-0.00900251331109419)
--(axis cs:-0.0343062384783924,-0.00763708505866078)
--cycle;
\path [fill=dodgerblue19129213]
(axis cs:-0.0444449687170643,-0.0168940739524316)
--(axis cs:-0.0398697962924837,-0.0182843546086722)
--(axis cs:-0.0370815239461359,-0.0129482930306274)
--(axis cs:-0.0416525345231005,-0.0115736760995666)
--cycle;
\path [fill=dodgerblue14119219]
(axis cs:-0.0518084158359676,-0.020839856128277)
--(axis cs:-0.0472504700377454,-0.0222393684711048)
--(axis cs:-0.0444449687170643,-0.0168940739524316)
--(axis cs:-0.0489988329058464,-0.0155102683369765)
--cycle;
\path [fill=dodgerblue3108224]
(axis cs:-0.0591718653075751,-0.0247856394232885)
--(axis cs:-0.0546311461463509,-0.0261943834583873)
--(axis cs:-0.0518084158359676,-0.020839856128277)
--(axis cs:-0.0563451336256558,-0.0194468618225809)
--cycle;
\path [fill=yellow24724417]
(axis cs:0.0252350435420258,0.0514321218769064)
--(axis cs:0.029971672177443,0.0502021021987509)
--(axis cs:0.0326213754726831,0.0567653375229088)
--(axis cs:0.0278859379793517,0.0579781543483271)
--cycle;
\path [fill=sandybrown23618576]
(axis cs:0.00781874058273858,0.0342176608447215)
--(axis cs:0.0125146583730791,0.0329473738280257)
--(axis cs:0.0151915788908389,0.0395375298134116)
--(axis cs:0.010496725544633,0.0407904885720435)
--cycle;
\path [fill=lightseagreen6156207]
(axis cs:-0.0297180815240209,-0.00900251331109419)
--(axis cs:-0.0251084558795504,-0.0103743306117435)
--(axis cs:-0.0223685386723843,-0.00398311342781738)
--(axis cs:-0.0269766659215075,-0.00262853434994103)
--cycle;
\path [fill=darkseagreen138190117]
(axis cs:-0.00960220158643413,0.0178181491836184)
--(axis cs:-0.00494483133805925,0.0165088891699653)
--(axis cs:-0.00223908018739899,0.0228549544059173)
--(axis cs:-0.00689479054031263,0.0241471858997253)
--cycle;
\path [fill=goldenrod25419753]
(axis cs:0.0151915788908389,0.0395375298134116)
--(axis cs:0.0199084101285282,0.0382787056124326)
--(axis cs:0.0225758805010002,0.04486567038954)
--(axis cs:0.0178601691059375,0.0461071781690331)
--cycle;
\path [fill=darkkhaki174189103]
(axis cs:-0.00223908018739899,0.0228549544059173)
--(axis cs:0.0024384317138063,0.0215566717093539)
--(axis cs:0.00513320076432047,0.0278980252530939)
--(axis cs:0.000457409867863144,0.0291792986364568)
--cycle;
\path [fill=lightseagreen56185157]
(axis cs:-0.0196838077642421,0.00643215917260348)
--(axis cs:-0.0150449500564563,0.00509479300575062)
--(axis cs:-0.0123189054495317,0.0114673888880697)
--(axis cs:-0.0169561804919516,0.0127875979321411)
--cycle;
\path [fill=gold24621836]
(axis cs:0.0225758805010002,0.04486567038954)
--(axis cs:0.0273136846315111,0.0436183462441519)
--(axis cs:0.029971672177443,0.0502021021987509)
--(axis cs:0.0252350435420258,0.0514321218769064)
--cycle;
\path [fill=darkkhaki20618790]
(axis cs:0.00513320076432047,0.0278980252530939)
--(axis cs:0.00983090089202801,0.0266107482405115)
--(axis cs:0.0125146583730791,0.0329473738280257)
--(axis cs:0.00781874058273858,0.0342176608447215)
--cycle;
\path [fill=lightseagreen13173186]
(axis cs:-0.0223685386723843,-0.00398311342781738)
--(axis cs:-0.0177388358566118,-0.00534403473749343)
--(axis cs:-0.0150449500564563,0.00509479300575062)
--(axis cs:-0.0196838077642421,0.00643215917260348)
--cycle;
\path [fill=mediumaquamarine98190134]
(axis cs:-0.0123189054495317,0.0114673888880697)
--(axis cs:-0.00765989137690144,0.0101409908506714)
--(axis cs:-0.00494483133805925,0.0165088891699653)
--(axis cs:-0.00960220158643413,0.0178181491836184)
--cycle;
\path [fill=dodgerblue16142210]
(axis cs:-0.0324891249061847,-0.0143293420061684)
--(axis cs:-0.0278751869313344,-0.0157168682768483)
--(axis cs:-0.0251084558795504,-0.0103743306117435)
--(axis cs:-0.0297180815240209,-0.00900251331109419)
--cycle;
\path [fill=dodgerblue19129213]
(axis cs:-0.0398697962924837,-0.0182843546086722)
--(axis cs:-0.0352731657245813,-0.0196811558588512)
--(axis cs:-0.0324891249061847,-0.0143293420061684)
--(axis cs:-0.0370815239461359,-0.0129482930306274)
--cycle;
\path [fill=dodgerblue14119219]
(axis cs:-0.0472504700377454,-0.0222393684711048)
--(axis cs:-0.0426711468878563,-0.0236454447067122)
--(axis cs:-0.0398697962924837,-0.0182843546086722)
--(axis cs:-0.0444449687170643,-0.0168940739524316)
--cycle;
\path [fill=dodgerblue3108224]
(axis cs:-0.0546311461463509,-0.0261943834583873)
--(axis cs:-0.0500691304251886,-0.0276097346851479)
--(axis cs:-0.0472504700377454,-0.0222393684711048)
--(axis cs:-0.0518084158359676,-0.020839856128277)
--cycle;
\path [fill=royalblue1392221]
(axis cs:-0.0620118246072606,-0.0301493999240806)
--(axis cs:-0.0574671163264609,-0.0315740261482549)
--(axis cs:-0.0546311461463509,-0.0261943834583873)
--(axis cs:-0.0591718653075751,-0.0247856394232885)
--cycle;
\path [fill=yellow24724417]
(axis cs:0.029971672177443,0.0502021021987509)
--(axis cs:0.0347305088489212,0.0489663154817432)
--(axis cs:0.0373789807367738,0.0555468431977038)
--(axis cs:0.0326213754726831,0.0567653375229088)
--cycle;
\path [fill=sandybrown23618576]
(axis cs:0.0125146583730791,0.0329473738280257)
--(axis cs:0.0172325934293032,0.031671130947483)
--(axis cs:0.0199084101285282,0.0382787056124326)
--(axis cs:0.0151915788908389,0.0395375298134116)
--cycle;
\path [fill=lightseagreen6156207]
(axis cs:-0.0251084558795504,-0.0103743306117435)
--(axis cs:-0.0204772105086793,-0.0117525819087733)
--(axis cs:-0.0177388358566118,-0.00534403473749343)
--(axis cs:-0.0223685386723843,-0.00398311342781738)
--cycle;
\path [fill=darkseagreen138190117]
(axis cs:-0.00494483133805925,0.0165088891699653)
--(axis cs:-0.000265614228611591,0.0151934876601241)
--(axis cs:0.0024384317138063,0.0215566717093539)
--(axis cs:-0.00223908018739899,0.0228549544059173)
--cycle;
\path [fill=goldenrod25419753]
(axis cs:0.0199084101285282,0.0382787056124326)
--(axis cs:0.0246473739465456,0.0370139746858695)
--(axis cs:0.0273136846315111,0.0436183462441519)
--(axis cs:0.0225758805010002,0.04486567038954)
--cycle;
\path [fill=darkkhaki174189103]
(axis cs:0.0024384317138063,0.0215566717093539)
--(axis cs:0.00713789865964159,0.0202522952058377)
--(axis cs:0.00983090089202801,0.0266107482405115)
--(axis cs:0.00513320076432047,0.0278980252530939)
--cycle;
\path [fill=lightseagreen56185157]
(axis cs:-0.0150449500564563,0.00509479300575062)
--(axis cs:-0.0103843084263092,0.00375114661161284)
--(axis cs:-0.00765989137690144,0.0101409908506714)
--(axis cs:-0.0123189054495317,0.0114673888880697)
--cycle;
\path [fill=gold24621836]
(axis cs:0.0273136846315111,0.0436183462441519)
--(axis cs:0.0320737371160816,0.0423651647630764)
--(axis cs:0.0347305088489212,0.0489663154817432)
--(axis cs:0.029971672177443,0.0502021021987509)
--cycle;
\path [fill=darkkhaki20618790]
(axis cs:0.00983090089202801,0.0266107482405115)
--(axis cs:0.0145506646018733,0.0253174253030447)
--(axis cs:0.0172325934293032,0.031671130947483)
--(axis cs:0.0125146583730791,0.0329473738280257)
--cycle;
\path [fill=lightseagreen13173186]
(axis cs:-0.0177388358566118,-0.00534403473749343)
--(axis cs:-0.0130874055912109,-0.00671134292563249)
--(axis cs:-0.0103843084263092,0.00375114661161284)
--(axis cs:-0.0150449500564563,0.00509479300575062)
--cycle;
\path [fill=mediumaquamarine98190134]
(axis cs:-0.00765989137690144,0.0101409908506714)
--(axis cs:-0.00297898504967398,0.00880836019772842)
--(axis cs:-0.000265614228611591,0.0151934876601241)
--(axis cs:-0.00494483133805925,0.0165088891699653)
--cycle;
\path [fill=dodgerblue16142210]
(axis cs:-0.0278751869313344,-0.0157168682768483)
--(axis cs:-0.0232395581348896,-0.0171109175187859)
--(axis cs:-0.0204772105086793,-0.0117525819087733)
--(axis cs:-0.0251084558795504,-0.0103743306117435)
--cycle;
\path [fill=dodgerblue19129213]
(axis cs:-0.0352731657245813,-0.0196811558588512)
--(axis cs:-0.0306549256963577,-0.0210845236844264)
--(axis cs:-0.0278751869313344,-0.0157168682768483)
--(axis cs:-0.0324891249061847,-0.0143293420061684)
--cycle;
\path [fill=dodgerblue14119219]
(axis cs:-0.0426711468878563,-0.0236454447067122)
--(axis cs:-0.0380702956389974,-0.025058131121896)
--(axis cs:-0.0352731657245813,-0.0196811558588512)
--(axis cs:-0.0398697962924837,-0.0182843546086722)
--cycle;
\path [fill=dodgerblue3108224]
(axis cs:-0.0500691304251886,-0.0276097346851479)
--(axis cs:-0.0454856679664828,-0.0290317396957067)
--(axis cs:-0.0426711468878563,-0.0236454447067122)
--(axis cs:-0.0472504700377454,-0.0222393684711048)
--cycle;
\path [fill=royalblue1392221]
(axis cs:-0.0574671163264609,-0.0315740261482549)
--(axis cs:-0.0529010426696272,-0.0330053497604917)
--(axis cs:-0.0500691304251886,-0.0276097346851479)
--(axis cs:-0.0546311461463509,-0.0261943834583873)
--cycle;
\path [fill=royalblue4770191]
(axis cs:-0.0648651045988038,-0.0355383188349988)
--(axis cs:-0.0603164197548744,-0.0369789610548204)
--(axis cs:-0.0574671163264609,-0.0315740261482549)
--(axis cs:-0.0620118246072606,-0.0301493999240806)
--cycle;
\path [fill=yellow24724417]
(axis cs:0.0347305088489212,0.0489663154817432)
--(axis cs:0.0395117101095122,0.0477247210717919)
--(axis cs:0.0421589097961623,0.0543226314124791)
--(axis cs:0.0373789807367738,0.0555468431977038)
--cycle;
\path [fill=sandybrown23618576]
(axis cs:0.0172325934293032,0.031671130947483)
--(axis cs:0.0219727009603963,0.0303888902176994)
--(axis cs:0.0246473739465456,0.0370139746858695)
--(axis cs:0.0199084101285282,0.0382787056124326)
--cycle;
\path [fill=lightseagreen6156207]
(axis cs:-0.0204772105086793,-0.0117525819087733)
--(axis cs:-0.015824192955017,-0.0131373125729642)
--(axis cs:-0.0130874055912109,-0.00671134292563249)
--(axis cs:-0.0177388358566118,-0.00534403473749343)
--cycle;
\path [fill=darkseagreen138190117]
(axis cs:-0.000265614228611591,0.0151934876601241)
--(axis cs:0.0044356038226585,0.0138719013395734)
--(axis cs:0.00713789865964159,0.0202522952058377)
--(axis cs:0.0024384317138063,0.0215566717093539)
--cycle;
\path [fill=goldenrod25419753]
(axis cs:0.0246473739465456,0.0370139746858695)
--(axis cs:0.0294086264888283,0.0357432953621516)
--(axis cs:0.0320737371160816,0.0423651647630764)
--(axis cs:0.0273136846315111,0.0436183462441519)
--cycle;
\path [fill=darkkhaki174189103]
(axis cs:0.00713789865964159,0.0202522952058377)
--(axis cs:0.0118594755907797,0.0189417818902841)
--(axis cs:0.0145506646018733,0.0253174253030447)
--(axis cs:0.00983090089202801,0.0266107482405115)
--cycle;
\path [fill=lightseagreen56185157]
(axis cs:-0.0103843084263092,0.00375114661161284)
--(axis cs:-0.00570172906779864,0.00240117564846711)
--(axis cs:-0.00297898504967398,0.00880836019772842)
--(axis cs:-0.00765989137690144,0.0101409908506714)
--cycle;
\path [fill=gold24621836]
(axis cs:0.0320737371160816,0.0423651647630764)
--(axis cs:0.0368561950383086,0.0411060845908336)
--(axis cs:0.0395117101095122,0.0477247210717919)
--(axis cs:0.0347305088489212,0.0489663154817432)
--cycle;
\path [fill=darkkhaki20618790]
(axis cs:0.0145506646018733,0.0253174253030447)
--(axis cs:0.0192926476980297,0.0240180137467954)
--(axis cs:0.0219727009603963,0.0303888902176994)
--(axis cs:0.0172325934293032,0.031671130947483)
--cycle;
\path [fill=lightseagreen13173186]
(axis cs:-0.0130874055912109,-0.00671134292563249)
--(axis cs:-0.00841409456425918,-0.00808508305894201)
--(axis cs:-0.00570172906779864,0.00240117564846711)
--(axis cs:-0.0103843084263092,0.00375114661161284)
--cycle;
\path [fill=mediumaquamarine98190134]
(axis cs:-0.00297898504967398,0.00880836019772842)
--(axis cs:0.00172396819990219,0.00746945289610438)
--(axis cs:0.0044356038226585,0.0138719013395734)
--(axis cs:-0.000265614228611591,0.0151934876601241)
--cycle;
\path [fill=dodgerblue16142210]
(axis cs:-0.0232395581348896,-0.0171109175187859)
--(axis cs:-0.0185820851987053,-0.0185115358385728)
--(axis cs:-0.015824192955017,-0.0131373125729642)
--(axis cs:-0.0204772105086793,-0.0117525819087733)
--cycle;
\path [fill=dodgerblue19129213]
(axis cs:-0.0306549256963577,-0.0210845236844264)
--(axis cs:-0.0260149234647329,-0.0224945045002056)
--(axis cs:-0.0232395581348896,-0.0171109175187859)
--(axis cs:-0.0278751869313344,-0.0157168682768483)
--cycle;
\path [fill=dodgerblue14119219]
(axis cs:-0.0380702956389974,-0.025058131121896)
--(axis cs:-0.0334477641231541,-0.026477474439681)
--(axis cs:-0.0306549256963577,-0.0210845236844264)
--(axis cs:-0.0352731657245813,-0.0196811558588512)
--cycle;
\path [fill=dodgerblue3108224]
(axis cs:-0.0454856679664828,-0.0290317396957067)
--(axis cs:-0.0408806071772848,-0.0304604455213056)
--(axis cs:-0.0380702956389974,-0.025058131121896)
--(axis cs:-0.0426711468878563,-0.0236454447067122)
--cycle;
\path [fill=royalblue1392221]
(axis cs:-0.0529010426696272,-0.0330053497604917)
--(axis cs:-0.0483134526188782,-0.0344434181002505)
--(axis cs:-0.0454856679664828,-0.0290317396957067)
--(axis cs:-0.0500691304251886,-0.0276097346851479)
--cycle;
\path [fill=royalblue4770191]
(axis cs:-0.0603164197548744,-0.0369789610548204)
--(axis cs:-0.0557463004536839,-0.0384263919146882)
--(axis cs:-0.0529010426696272,-0.0330053497604917)
--(axis cs:-0.0574671163264609,-0.0315740261482549)
--cycle;
\path [fill=darkslateblue5454159]
(axis cs:-0.067731799222609,-0.040952573584519)
--(axis cs:-0.0631791506821041,-0.0424093669704544)
--(axis cs:-0.0603164197548744,-0.0369789610548204)
--(axis cs:-0.0648651045988038,-0.0355383188349988)
--cycle;
\path [fill=yellow24724417]
(axis cs:0.0395117101095122,0.0477247210717919)
--(axis cs:0.0443154339872036,0.0464772779317903)
--(axis cs:0.0469613201430371,0.0530926618311142)
--(axis cs:0.0421589097961623,0.0543226314124791)
--cycle;
\path [fill=sandybrown23618576]
(axis cs:0.0219727009603963,0.0303888902176994)
--(axis cs:0.0267351376376241,0.0291006092577212)
--(axis cs:0.0294086264888283,0.0357432953621516)
--(axis cs:0.0246473739465456,0.0370139746858695)
--cycle;
\path [fill=lightseagreen6156207]
(axis cs:-0.015824192955017,-0.0131373125729642)
--(axis cs:-0.0111492493253549,-0.0145285684026917)
--(axis cs:-0.00841409456425918,-0.00808508305894201)
--(axis cs:-0.0130874055912109,-0.00671134292563249)
--cycle;
\path [fill=darkseagreen138190117]
(axis cs:0.0044356038226585,0.0138719013395734)
--(axis cs:0.00915897834884149,0.0125440864855156)
--(axis cs:0.0118594755907797,0.0189417818902841)
--(axis cs:0.00713789865964159,0.0202522952058377)
--cycle;
\path [fill=goldenrod25419753]
(axis cs:0.0294086264888283,0.0357432953621516)
--(axis cs:0.0341923253715569,0.0344666255767969)
--(axis cs:0.0368561950383086,0.0411060845908336)
--(axis cs:0.0320737371160816,0.0423651647630764)
--cycle;
\path [fill=darkkhaki174189103]
(axis cs:0.0118594755907797,0.0189417818902841)
--(axis cs:0.0166033189092567,0.0176250883519951)
--(axis cs:0.0192926476980297,0.0240180137467954)
--(axis cs:0.0145506646018733,0.0253174253030447)
--cycle;
\path [fill=lightseagreen56185157]
(axis cs:-0.00570172906779864,0.00240117564846711)
--(axis cs:-0.000997056723570961,0.001044835356171)
--(axis cs:0.00172396819990219,0.00746945289610438)
--(axis cs:-0.00297898504967398,0.00880836019772842)
--cycle;
\path [fill=gold24621836]
(axis cs:0.0368561950383086,0.0411060845908336)
--(axis cs:0.0416612169640534,0.0398410639817071)
--(axis cs:0.0443154339872036,0.0464772779317903)
--(axis cs:0.0395117101095122,0.0477247210717919)
--cycle;
\path [fill=darkkhaki20618790]
(axis cs:0.0192926476980297,0.0240180137467954)
--(axis cs:0.0240570074551002,0.0227124704749342)
--(axis cs:0.0267351376376241,0.0291006092577212)
--(axis cs:0.0219727009603963,0.0303888902176994)
--cycle;
\path [fill=lightseagreen13173186]
(axis cs:-0.00841409456425918,-0.00808508305894201)
--(axis cs:-0.00371874801804615,-0.00946530062912502)
--(axis cs:-0.000997056723570961,0.001044835356171)
--(axis cs:-0.00570172906779864,0.00240117564846711)
--cycle;
\path [fill=mediumaquamarine98190134]
(axis cs:0.00172396819990219,0.00746945289610438)
--(axis cs:0.00644912449997797,0.00612422449689445)
--(axis cs:0.00915897834884149,0.0125440864855156)
--(axis cs:0.0044356038226585,0.0138719013395734)
--cycle;
\path [fill=yellow24724417]
(axis cs:0.0443154339872036,0.0464772779317903)
--(axis cs:0.0491418400023313,0.045223944637095)
--(axis cs:0.0517863707545354,0.0518568937371714)
--(axis cs:0.0469613201430371,0.0530926618311142)
--cycle;
\path [fill=sandybrown23618576]
(axis cs:0.0267351376376241,0.0291006092577212)
--(axis cs:0.0315200616117948,0.0278062452863654)
--(axis cs:0.0341923253715569,0.0344666255767969)
--(axis cs:0.0294086264888283,0.0357432953621516)
--cycle;
\path [fill=dodgerblue14144209]
(axis cs:-0.0185820851987053,-0.0185115358385728)
--(axis cs:-0.013926688524209,-0.0169538692044517)
--(axis cs:-0.0111492493253549,-0.0145285684026917)
--(axis cs:-0.015824192955017,-0.0131373125729642)
--cycle;
\path [fill=dodgerblue20132211]
(axis cs:-0.0260149234647329,-0.0224945045002056)
--(axis cs:-0.021389981866474,-0.0209531577638003)
--(axis cs:-0.0185820851987053,-0.0185115358385728)
--(axis cs:-0.0232395581348896,-0.0171109175187859)
--cycle;
\path [fill=dodgerblue16122217]
(axis cs:-0.0334477641231541,-0.026477474439681)
--(axis cs:-0.0288532776211546,-0.0249524475592775)
--(axis cs:-0.0260149234647329,-0.0224945045002056)
--(axis cs:-0.0306549256963577,-0.0210845236844264)
--cycle;
\path [fill=dodgerblue6111222]
(axis cs:-0.0408806071772848,-0.0304604455213056)
--(axis cs:-0.0363165757878822,-0.028951738643114)
--(axis cs:-0.0334477641231541,-0.026477474439681)
--(axis cs:-0.0380702956389974,-0.025058131121896)
--cycle;
\path [fill=dodgerblue498224]
(axis cs:-0.0483134526188782,-0.0344434181002505)
--(axis cs:-0.0437798763702184,-0.0329510308792601)
--(axis cs:-0.0408806071772848,-0.0304604455213056)
--(axis cs:-0.0454856679664828,-0.0290317396957067)
--cycle;
\path [fill=royalblue4175201]
(axis cs:-0.0557463004536839,-0.0384263919146882)
--(axis cs:-0.0512431793592767,-0.036950324623821)
--(axis cs:-0.0483134526188782,-0.0344434181002505)
--(axis cs:-0.0529010426696272,-0.0330053497604917)
--cycle;
\path [fill=darkslateblue5357165]
(axis cs:-0.0631791506821041,-0.0424093669704544)
--(axis cs:-0.0587064847612751,-0.0409496196142794)
--(axis cs:-0.0557463004536839,-0.0384263919146882)
--(axis cs:-0.0603164197548744,-0.0369789610548204)
--cycle;
\path [fill=darkslateblue5342134]
(axis cs:-0.0706120033129749,-0.0463923430696876)
--(axis cs:-0.0661697925766106,-0.0449489158564882)
--(axis cs:-0.0631791506821041,-0.0424093669704544)
--(axis cs:-0.067731799222609,-0.040952573584519)
--cycle;
\path [fill=darkseagreen138190117]
(axis cs:0.00915897834884149,0.0125440864855156)
--(axis cs:0.0139046663525217,0.0112099989620553)
--(axis cs:0.0166033189092567,0.0176250883519951)
--(axis cs:0.0118594755907797,0.0189417818902841)
--cycle;
\path [fill=goldenrod25419753]
(axis cs:0.0341923253715569,0.0344666255767969)
--(axis cs:0.038998629700547,0.0331839228677707)
--(axis cs:0.0416612169640534,0.0398410639817071)
--(axis cs:0.0368561950383086,0.0411060845908336)
--cycle;
\path [fill=lightseagreen6160204]
(axis cs:-0.0111492493253549,-0.0145285684026917)
--(axis cs:-0.00646747022023842,-0.0118713687654844)
--(axis cs:-0.00371874801804615,-0.00946530062912502)
--(axis cs:-0.00841409456425918,-0.00808508305894201)
--cycle;
\path [fill=darkkhaki174189103]
(axis cs:0.0166033189092567,0.0176250883519951)
--(axis cs:0.0213695864957414,0.0163021707698657)
--(axis cs:0.0240570074551002,0.0227124704749342)
--(axis cs:0.0192926476980297,0.0240180137467954)
--cycle;
\path [fill=lightseagreen56185157]
(axis cs:-0.000997056723570961,0.001044835356171)
--(axis cs:0.00372986533223818,-0.000317919448784454)
--(axis cs:0.00644912449997797,0.00612422449689445)
--(axis cs:0.00172396819990219,0.00746945289610438)
--cycle;
\path [fill=gold24621836]
(axis cs:0.0416612169640534,0.0398410639817071)
--(axis cs:0.0464889629589679,0.0385700607951301)
--(axis cs:0.0491418400023313,0.045223944637095)
--(axis cs:0.0443154339872036,0.0464772779317903)
--cycle;
\path [fill=darkkhaki20618790]
(axis cs:0.0240570074551002,0.0227124704749342)
--(axis cs:0.028843902635506,0.0214007519829356)
--(axis cs:0.0315200616117948,0.0278062452863654)
--(axis cs:0.0267351376376241,0.0291006092577212)
--cycle;
\path [fill=mediumaquamarine98190134]
(axis cs:0.00644912449997797,0.00612422449689445)
--(axis cs:0.0111966414563818,0.00477263013050649)
--(axis cs:0.0139046663525217,0.0112099989620553)
--(axis cs:0.00915897834884149,0.0125440864855156)
--cycle;
\path [fill=yellow24724417]
(axis cs:0.0491418400023313,0.045223944637095)
--(axis cs:0.0539910891852363,0.0439646793709406)
--(axis cs:0.0566342221102846,0.0506152860294023)
--(axis cs:0.0517863707545354,0.0518568937371714)
--cycle;
\path [fill=sandybrown23618576]
(axis cs:0.0315200616117948,0.0278062452863654)
--(axis cs:0.0363276325307653,0.026505755117484)
--(axis cs:0.038998629700547,0.0331839228677707)
--(axis cs:0.0341923253715569,0.0344666255767969)
--cycle;
\path [fill=darkseagreen138190117]
(axis cs:0.0139046663525217,0.0112099989620553)
--(axis cs:0.0186728263231727,0.00986959421530944)
--(axis cs:0.0213695864957414,0.0163021707698657)
--(axis cs:0.0166033189092567,0.0176250883519951)
--cycle;
\path [fill=goldenrod25419753]
(axis cs:0.038998629700547,0.0331839228677707)
--(axis cs:0.0438277000888907,0.0318951443707779)
--(axis cs:0.0464889629589679,0.0385700607951301)
--(axis cs:0.0416612169640534,0.0398410639817071)
--cycle;
\path [fill=lightseagreen24177177]
(axis cs:-0.00371874801804615,-0.00946530062912502)
--(axis cs:0.001003524515317,-0.00269359139541891)
--(axis cs:0.00372986533223818,-0.000317919448784454)
--(axis cs:-0.000997056723570961,0.001044835356171)
--cycle;
\path [fill=darkkhaki174189103]
(axis cs:0.0213695864957414,0.0163021707698657)
--(axis cs:0.0261584377270507,0.0149729849075222)
--(axis cs:0.028843902635506,0.0214007519829356)
--(axis cs:0.0240570074551002,0.0227124704749342)
--cycle;
\path [fill=gold24621836]
(axis cs:0.0464889629589679,0.0385700607951301)
--(axis cs:0.0513395946062674,0.0372930324910056)
--(axis cs:0.0539910891852363,0.0439646793709406)
--(axis cs:0.0491418400023313,0.045223944637095)
--cycle;
\path [fill=darkkhaki20618790]
(axis cs:0.028843902635506,0.0214007519829356)
--(axis cs:0.0336534935071209,0.0200828143537456)
--(axis cs:0.0363276325307653,0.026505755117484)
--(axis cs:0.0315200616117948,0.0278062452863654)
--cycle;
\path [fill=dodgerblue16142210]
(axis cs:-0.021389981866474,-0.0209531577638003)
--(axis cs:-0.0167269832158048,-0.019399127758984)
--(axis cs:-0.013926688524209,-0.0169538692044517)
--(axis cs:-0.0185820851987053,-0.0185115358385728)
--cycle;
\path [fill=dodgerblue19129213]
(axis cs:-0.0288532776211546,-0.0249524475592775)
--(axis cs:-0.0242209846646473,-0.0234148717512646)
--(axis cs:-0.021389981866474,-0.0209531577638003)
--(axis cs:-0.0260149234647329,-0.0224945045002056)
--cycle;
\path [fill=dodgerblue14119219]
(axis cs:-0.0363165757878822,-0.028951738643114)
--(axis cs:-0.0317149885458886,-0.0274306169901804)
--(axis cs:-0.0288532776211546,-0.0249524475592775)
--(axis cs:-0.0334477641231541,-0.026477474439681)
--cycle;
\path [fill=dodgerblue3108224]
(axis cs:-0.0437798763702184,-0.0329510308792601)
--(axis cs:-0.0392089948590733,-0.0314463635281024)
--(axis cs:-0.0363165757878822,-0.028951738643114)
--(axis cs:-0.0408806071772848,-0.0304604455213056)
--cycle;
\path [fill=royalblue1392221]
(axis cs:-0.0512431793592767,-0.036950324623821)
--(axis cs:-0.0467030036080131,-0.0354621112286232)
--(axis cs:-0.0437798763702184,-0.0329510308792601)
--(axis cs:-0.0483134526188782,-0.0344434181002505)
--cycle;
\path [fill=royalblue4770191]
(axis cs:-0.0587064847612751,-0.0409496196142794)
--(axis cs:-0.0541970147831708,-0.0394778604487855)
--(axis cs:-0.0512431793592767,-0.036950324623821)
--(axis cs:-0.0557463004536839,-0.0384263919146882)
--cycle;
\path [fill=darkslateblue5454159]
(axis cs:-0.0661697925766106,-0.0449489158564882)
--(axis cs:-0.061691028391241,-0.04349361092538)
--(axis cs:-0.0587064847612751,-0.0409496196142794)
--(axis cs:-0.0631791506821041,-0.0424093669704544)
--cycle;
\path [fill=lightseagreen6156207]
(axis cs:-0.013926688524209,-0.0169538692044517)
--(axis cs:-0.00923882591106275,-0.014297248913319)
--(axis cs:-0.00646747022023842,-0.0118713687654844)
--(axis cs:-0.0111492493253549,-0.0145285684026917)
--cycle;
\path [fill=mediumseagreen67186151]
(axis cs:0.00372986533223818,-0.000317919448784454)
--(axis cs:0.00849930363170488,0.00241640442376586)
--(axis cs:0.0111966414563818,0.00477263013050649)
--(axis cs:0.00644912449997797,0.00612422449689445)
--cycle;
\path [fill=yellow24724417]
(axis cs:0.0539910891852363,0.0439646793709406)
--(axis cs:0.0588633440941742,0.0426994399197891)
--(axis cs:0.0615050362101951,0.0493677972171913)
--(axis cs:0.0566342221102846,0.0506152860294023)
--cycle;
\path [fill=sandybrown23618576]
(axis cs:0.0363276325307653,0.026505755117484)
--(axis cs:0.0411580115571978,0.0251990951551612)
--(axis cs:0.0438277000888907,0.0318951443707779)
--(axis cs:0.038998629700547,0.0331839228677707)
--cycle;
\path [fill=darkseagreen107190130]
(axis cs:0.0111966414563818,0.00477263013050649)
--(axis cs:0.0160045669299443,0.0075328657663582)
--(axis cs:0.0186728263231727,0.00986959421530944)
--(axis cs:0.0139046663525217,0.0112099989620553)
--cycle;
\path [fill=goldenrod25419753]
(axis cs:0.0438277000888907,0.0318951443707779)
--(axis cs:0.0486796986748458,0.0306002468144878)
--(axis cs:0.0513395946062674,0.0372930324910056)
--(axis cs:0.0464889629589679,0.0385700607951301)
--cycle;
\path [fill=darkseagreen148190113]
(axis cs:0.0186728263231727,0.00986959421530944)
--(axis cs:0.0235193324211067,0.012655804881048)
--(axis cs:0.0261584377270507,0.0149729849075222)
--(axis cs:0.0213695864957414,0.0163021707698657)
--cycle;
\path [fill=lightseagreen13173186]
(axis cs:-0.00646747022023842,-0.0118713687654844)
--(axis cs:-0.00174533316461829,-0.00508888402904051)
--(axis cs:0.001003524515317,-0.00269359139541891)
--(axis cs:-0.00371874801804615,-0.00946530062912502)
--cycle;
\path [fill=gold24621836]
(axis cs:0.0513395946062674,0.0372930324910056)
--(axis cs:0.056213275024759,0.036009936124961)
--(axis cs:0.0588633440941742,0.0426994399197891)
--(axis cs:0.0539910891852363,0.0439646793709406)
--cycle;
\path [fill=darkkhaki18218999]
(axis cs:0.0261584377270507,0.0149729849075222)
--(axis cs:0.0310436181616382,0.0177852340617641)
--(axis cs:0.0336534935071209,0.0200828143537456)
--(axis cs:0.028843902635506,0.0214007519829356)
--cycle;
\path [fill=darkkhaki21418686]
(axis cs:0.0336534935071209,0.0200828143537456)
--(axis cs:0.0385774422628441,0.022921166058712)
--(axis cs:0.0411580115571978,0.0251990951551612)
--(axis cs:0.0363276325307653,0.026505755117484)
--cycle;
\path [fill=yellow24724417]
(axis cs:0.0588633440941742,0.0426994399197891)
--(axis cs:0.0637587688334809,0.0414281836686124)
--(axis cs:0.0663989765925056,0.0481143854159336)
--(axis cs:0.0615050362101951,0.0493677972171913)
--cycle;
\path [fill=sandybrown24318571]
(axis cs:0.0411580115571978,0.0251990951551612)
--(axis cs:0.0461281388044721,0.0283428118061872)
--(axis cs:0.0486796986748458,0.0306002468144878)
--(axis cs:0.0438277000888907,0.0318951443707779)
--cycle;
\path [fill=lightseagreen56185157]
(axis cs:0.001003524515317,-0.00269359139541891)
--(axis cs:0.00577967433717615,4.07062824547119e-05)
--(axis cs:0.00849930363170488,0.00241640442376586)
--(axis cs:0.00372986533223818,-0.000317919448784454)
--cycle;
\path [fill=dodgerblue16142210]
(axis cs:-0.0242209846646473,-0.0234148717512646)
--(axis cs:-0.0195504166818047,-0.0218645914320071)
--(axis cs:-0.0167269832158048,-0.019399127758984)
--(axis cs:-0.021389981866474,-0.0209531577638003)
--cycle;
\path [fill=dodgerblue19129213]
(axis cs:-0.0317149885458886,-0.0274306169901804)
--(axis cs:-0.0270753799814548,-0.0258969268314191)
--(axis cs:-0.0242209846646473,-0.0234148717512646)
--(axis cs:-0.0288532776211546,-0.0249524475592775)
--cycle;
\path [fill=dodgerblue14119219]
(axis cs:-0.0392089948590733,-0.0314463635281024)
--(axis cs:-0.0346003457337357,-0.0299292634881047)
--(axis cs:-0.0317149885458886,-0.0274306169901804)
--(axis cs:-0.0363165757878822,-0.028951738643114)
--cycle;
\path [fill=dodgerblue3108224]
(axis cs:-0.0467030036080131,-0.0354621112286232)
--(axis cs:-0.0421253139381035,-0.0339616014545757)
--(axis cs:-0.0392089948590733,-0.0314463635281024)
--(axis cs:-0.0437798763702184,-0.0329510308792601)
--cycle;
\path [fill=royalblue1392221]
(axis cs:-0.0541970147831708,-0.0394778604487855)
--(axis cs:-0.0496502845986236,-0.0379939405940657)
--(axis cs:-0.0467030036080131,-0.0354621112286232)
--(axis cs:-0.0512431793592767,-0.036950324623821)
--cycle;
\path [fill=royalblue4770191]
(axis cs:-0.061691028391241,-0.04349361092538)
--(axis cs:-0.0571752577050984,-0.0420262812645584)
--(axis cs:-0.0541970147831708,-0.0394778604487855)
--(axis cs:-0.0587064847612751,-0.0409496196142794)
--cycle;
\path [fill=lightseagreen6156207]
(axis cs:-0.0167269832158048,-0.019399127758984)
--(axis cs:-0.0120330957987385,-0.0167431867879295)
--(axis cs:-0.00923882591106275,-0.014297248913319)
--(axis cs:-0.013926688524209,-0.0169538692044517)
--cycle;
\path [fill=gold25320248]
(axis cs:0.0486796986748458,0.0306002468144878)
--(axis cs:0.0536908197489743,0.0337730630109281)
--(axis cs:0.056213275024759,0.036009936124961)
--(axis cs:0.0513395946062674,0.0372930324910056)
--cycle;
\path [fill=mediumaquamarine98190134]
(axis cs:0.00849930363170488,0.00241640442376586)
--(axis cs:0.0133142423702335,0.00517681374550034)
--(axis cs:0.0160045669299443,0.0075328657663582)
--(axis cs:0.0111966414563818,0.00477263013050649)
--cycle;
\path [fill=darkseagreen138190117]
(axis cs:0.0160045669299443,0.0075328657663582)
--(axis cs:0.0208583891657848,0.0103194507579571)
--(axis cs:0.0235193324211067,0.012655804881048)
--(axis cs:0.0186728263231727,0.00986959421530944)
--cycle;
\path [fill=gold24522332]
(axis cs:0.056213275024759,0.036009936124961)
--(axis cs:0.0612655136659433,0.039211940501248)
--(axis cs:0.0637587688334809,0.0414281836686124)
--(axis cs:0.0588633440941742,0.0426994399197891)
--cycle;
\path [fill=darkkhaki174189103]
(axis cs:0.0235193324211067,0.012655804881048)
--(axis cs:0.0284121330011606,0.0154686297634707)
--(axis cs:0.0310436181616382,0.0177852340617641)
--(axis cs:0.0261584377270507,0.0149729849075222)
--cycle;
\path [fill=yellow24825013]
(axis cs:0.0637587688334809,0.0414281836686124)
--(axis cs:0.0688522491531563,0.0446594634163324)
--(axis cs:0.0713162083520864,0.0468550083423477)
--(axis cs:0.0663989765925056,0.0481143854159336)
--cycle;
\path [fill=lightseagreen13173186]
(axis cs:-0.00923882591106275,-0.014297248913319)
--(axis cs:-0.00451698781460399,-0.00750404142860805)
--(axis cs:-0.00174533316461829,-0.00508888402904051)
--(axis cs:-0.00646747022023842,-0.0118713687654844)
--cycle;
\path [fill=darkkhaki20618790]
(axis cs:0.0310436181616382,0.0177852340617641)
--(axis cs:0.0359754922087058,0.0206243636636754)
--(axis cs:0.0385774422628441,0.022921166058712)
--(axis cs:0.0336534935071209,0.0200828143537456)
--cycle;
\path [fill=sandybrown23618576]
(axis cs:0.0385774422628441,0.022921166058712)
--(axis cs:0.0435554216926197,0.0260666584059931)
--(axis cs:0.0461281388044721,0.0283428118061872)
--(axis cs:0.0411580115571978,0.0251990951551612)
--cycle;
\path [fill=goldenrod25419753]
(axis cs:0.0461281388044721,0.0283428118061872)
--(axis cs:0.0511474318960244,0.031517627221268)
--(axis cs:0.0536908197489743,0.0337730630109281)
--(axis cs:0.0486796986748458,0.0306002468144878)
--cycle;
\path [fill=lightseagreen56185157]
(axis cs:-0.00174533316461829,-0.00508888402904051)
--(axis cs:0.0030374760928475,-0.00235470668256387)
--(axis cs:0.00577967433717615,4.07062824547119e-05)
--(axis cs:0.001003524515317,-0.00269359139541891)
--cycle;
\path [fill=dodgerblue16142210]
(axis cs:-0.0270753799814548,-0.0258969268314191)
--(axis cs:-0.0223972769047933,-0.0243505116941542)
--(axis cs:-0.0195504166818047,-0.0218645914320071)
--(axis cs:-0.0242209846646473,-0.0234148717512646)
--cycle;
\path [fill=dodgerblue19129213]
(axis cs:-0.0346003457337357,-0.0299292634881047)
--(axis cs:-0.0299534589576012,-0.028399576167258)
--(axis cs:-0.0270753799814548,-0.0258969268314191)
--(axis cs:-0.0317149885458886,-0.0274306169901804)
--cycle;
\path [fill=dodgerblue14119219]
(axis cs:-0.0421253139381035,-0.0339616014545757)
--(axis cs:-0.0375096434835243,-0.0324486419084078)
--(axis cs:-0.0346003457337357,-0.0299292634881047)
--(axis cs:-0.0392089948590733,-0.0314463635281024)
--cycle;
\path [fill=dodgerblue3108224]
(axis cs:-0.0496502845986236,-0.0379939405940657)
--(axis cs:-0.0450658304819289,-0.0364977089702567)
--(axis cs:-0.0421253139381035,-0.0339616014545757)
--(axis cs:-0.0467030036080131,-0.0354621112286232)
--cycle;
\path [fill=royalblue1392221]
(axis cs:-0.0571752577050984,-0.0420262812645584)
--(axis cs:-0.0526220199571387,-0.0405467772156781)
--(axis cs:-0.0496502845986236,-0.0379939405940657)
--(axis cs:-0.0541970147831708,-0.0394778604487855)
--cycle;
\path [fill=lightseagreen6156207]
(axis cs:-0.0195504166818047,-0.0218645914320071)
--(axis cs:-0.014850565252673,-0.0192094321847436)
--(axis cs:-0.0120330957987385,-0.0167431867879295)
--(axis cs:-0.0167269832158048,-0.019399127758984)
--cycle;
\path [fill=mediumaquamarine98190134]
(axis cs:0.00577967433717615,4.07062824547119e-05)
--(axis cs:0.0106015778058684,0.00280119746317773)
--(axis cs:0.0133142423702335,0.00517681374550034)
--(axis cs:0.00849930363170488,0.00241640442376586)
--cycle;
\path [fill=gold24621836]
(axis cs:0.0536908197489743,0.0337730630109281)
--(axis cs:0.0587515517371486,0.0369772911880742)
--(axis cs:0.0612655136659433,0.039211940501248)
--(axis cs:0.056213275024759,0.036009936124961)
--cycle;
\path [fill=darkseagreen138190117]
(axis cs:0.0133142423702335,0.00517681374550034)
--(axis cs:0.0181753357794511,0.00796368355807538)
--(axis cs:0.0208583891657848,0.0103194507579571)
--(axis cs:0.0160045669299443,0.0075328657663582)
--cycle;
\path [fill=yellow24724417]
(axis cs:0.0612655136659433,0.039211940501248)
--(axis cs:0.0663678101661283,0.0424456696918532)
--(axis cs:0.0688522491531563,0.0446594634163324)
--(axis cs:0.0637587688334809,0.0414281836686124)
--cycle;
\path [fill=darkkhaki174189103]
(axis cs:0.0208583891657848,0.0103194507579571)
--(axis cs:0.0257587685154256,0.0131327641979293)
--(axis cs:0.0284121330011606,0.0154686297634707)
--(axis cs:0.0235193324211067,0.012655804881048)
--cycle;
\path [fill=lightseagreen13173186]
(axis cs:-0.0120330957987385,-0.0167431867879295)
--(axis cs:-0.0073117242070184,-0.00993931173835424)
--(axis cs:-0.00451698781460399,-0.00750404142860805)
--(axis cs:-0.00923882591106275,-0.014297248913319)
--cycle;
\path [fill=darkkhaki20618790]
(axis cs:0.0284121330011606,0.0154686297634707)
--(axis cs:0.0333518945707975,0.0183084524382559)
--(axis cs:0.0359754922087058,0.0206243636636754)
--(axis cs:0.0310436181616382,0.0177852340617641)
--cycle;
\path [fill=sandybrown23618576]
(axis cs:0.0359754922087058,0.0206243636636754)
--(axis cs:0.0409612830936154,0.0237715528285102)
--(axis cs:0.0435554216926197,0.0260666584059931)
--(axis cs:0.0385774422628441,0.022921166058712)
--cycle;
\path [fill=goldenrod25419753]
(axis cs:0.0435554216926197,0.0260666584059931)
--(axis cs:0.0485828498169992,0.0292433967298998)
--(axis cs:0.0511474318960244,0.031517627221268)
--(axis cs:0.0461281388044721,0.0283428118061872)
--cycle;
\path [fill=lightseagreen56185157]
(axis cs:-0.00451698781460399,-0.00750404142860805)
--(axis cs:0.000272426794225699,-0.0047700809001401)
--(axis cs:0.0030374760928475,-0.00235470668256387)
--(axis cs:-0.00174533316461829,-0.00508888402904051)
--cycle;
\path [fill=dodgerblue16142210]
(axis cs:-0.0299534589576012,-0.028399576167258)
--(axis cs:-0.0252678566661961,-0.0268571442064765)
--(axis cs:-0.0223972769047933,-0.0243505116941542)
--(axis cs:-0.0270753799814548,-0.0258969268314191)
--cycle;
\path [fill=dodgerblue19129213]
(axis cs:-0.0375096434835243,-0.0324486419084078)
--(axis cs:-0.0328555175852581,-0.0309230771403951)
--(axis cs:-0.0299534589576012,-0.028399576167258)
--(axis cs:-0.0346003457337357,-0.0299292634881047)
--cycle;
\path [fill=dodgerblue14119219]
(axis cs:-0.0450658304819289,-0.0364977089702567)
--(axis cs:-0.0404431809981765,-0.0349890113532683)
--(axis cs:-0.0375096434835243,-0.0324486419084078)
--(axis cs:-0.0421253139381035,-0.0339616014545757)
--cycle;
\path [fill=dodgerblue3108224]
(axis cs:-0.0526220199571387,-0.0405467772156781)
--(axis cs:-0.0480308469042267,-0.0390549468978911)
--(axis cs:-0.0450658304819289,-0.0364977089702567)
--(axis cs:-0.0496502845986236,-0.0379939405940657)
--cycle;
\path [fill=lightseagreen6156207]
(axis cs:-0.0223972769047933,-0.0243505116941542)
--(axis cs:-0.0176915244006146,-0.021696239064358)
--(axis cs:-0.014850565252673,-0.0192094321847436)
--(axis cs:-0.0195504166818047,-0.0218645914320071)
--cycle;
\path [fill=mediumaquamarine98190134]
(axis cs:0.0030374760928475,-0.00235470668256387)
--(axis cs:0.00786629381522737,0.000405772215875058)
--(axis cs:0.0106015778058684,0.00280119746317773)
--(axis cs:0.00577967433717615,4.07062824547119e-05)
--cycle;
\path [fill=gold24621836]
(axis cs:0.0511474318960244,0.031517627221268)
--(axis cs:0.0562166240135154,0.0347240054753193)
--(axis cs:0.0587515517371486,0.0369772911880742)
--(axis cs:0.0536908197489743,0.0337730630109281)
--cycle;
\path [fill=darkseagreen138190117]
(axis cs:0.0106015778058684,0.00280119746317773)
--(axis cs:0.0154698955384146,0.00558826031322431)
--(axis cs:0.0181753357794511,0.00796368355807538)
--(axis cs:0.0133142423702335,0.00517681374550034)
--cycle;
\path [fill=yellow24724417]
(axis cs:0.0587515517371486,0.0369772911880742)
--(axis cs:0.0638626349914298,0.0402133987005247)
--(axis cs:0.0663678101661283,0.0424456696918532)
--(axis cs:0.0612655136659433,0.039211940501248)
--cycle;
\path [fill=darkkhaki174189103]
(axis cs:0.0181753357794511,0.00796368355807538)
--(axis cs:0.0230832506938078,0.0107773961423499)
--(axis cs:0.0257587685154256,0.0131327641979293)
--(axis cs:0.0208583891657848,0.0103194507579571)
--cycle;
\path [fill=lightseagreen13173186]
(axis cs:-0.014850565252673,-0.0192094321847436)
--(axis cs:-0.0101298318771176,-0.0123949472527764)
--(axis cs:-0.0073117242070184,-0.00993931173835424)
--(axis cs:-0.0120330957987385,-0.0167431867879295)
--cycle;
\path [fill=darkkhaki20618790]
(axis cs:0.0257587685154256,0.0131327641979293)
--(axis cs:0.0307063780667683,0.0159731929151536)
--(axis cs:0.0333518945707975,0.0183084524382559)
--(axis cs:0.0284121330011606,0.0154686297634707)
--cycle;
\path [fill=sandybrown23618576]
(axis cs:0.0333518945707975,0.0183084524382559)
--(axis cs:0.0383454543429563,0.021457257378883)
--(axis cs:0.0409612830936154,0.0237715528285102)
--(axis cs:0.0359754922087058,0.0206243636636754)
--cycle;
\path [fill=goldenrod25419753]
(axis cs:0.0409612830936154,0.0237715528285102)
--(axis cs:0.0459968074840673,0.0269501356275847)
--(axis cs:0.0485828498169992,0.0292433967298998)
--(axis cs:0.0435554216926197,0.0260666584059931)
--cycle;
\path [fill=lightseagreen56185157]
(axis cs:-0.0073117242070184,-0.00993931173835424)
--(axis cs:-0.00251576038447259,-0.00720566692334806)
--(axis cs:0.000272426794225699,-0.0047700809001401)
--(axis cs:-0.00451698781460399,-0.00750404142860805)
--cycle;
\path [fill=dodgerblue16142210]
(axis cs:-0.0328555175852581,-0.0309230771403951)
--(axis cs:-0.0281624536466546,-0.0293847489080931)
--(axis cs:-0.0252678566661961,-0.0268571442064765)
--(axis cs:-0.0299534589576012,-0.028399576167258)
--cycle;
\path [fill=dodgerblue19129213]
(axis cs:-0.0404431809981765,-0.0349890113532683)
--(axis cs:-0.0357818568095406,-0.0334676914393035)
--(axis cs:-0.0328555175852581,-0.0309230771403951)
--(axis cs:-0.0375096434835243,-0.0324486419084078)
--cycle;
\path [fill=dodgerblue14119219]
(axis cs:-0.0480308469042267,-0.0390549468978911)
--(axis cs:-0.0434012624872858,-0.0375506352605158)
--(axis cs:-0.0404431809981765,-0.0349890113532683)
--(axis cs:-0.0450658304819289,-0.0364977089702567)
--cycle;
\path [fill=lightseagreen6156207]
(axis cs:-0.0252678566661961,-0.0268571442064765)
--(axis cs:-0.0205562682282454,-0.0242038656397163)
--(axis cs:-0.0176915244006146,-0.021696239064358)
--(axis cs:-0.0223972769047933,-0.0243505116941542)
--cycle;
\path [fill=mediumaquamarine98190134]
(axis cs:0.000272426794225699,-0.0047700809001401)
--(axis cs:0.00510810629729068,-0.00200971079790599)
--(axis cs:0.00786629381522737,0.000405772215875058)
--(axis cs:0.0030374760928475,-0.00235470668256387)
--cycle;
\path [fill=gold24621836]
(axis cs:0.0485828498169992,0.0292433967298998)
--(axis cs:0.0536604671228237,0.0324518492526164)
--(axis cs:0.0562166240135154,0.0347240054753193)
--(axis cs:0.0511474318960244,0.031517627221268)
--cycle;
\path [fill=darkseagreen138190117]
(axis cs:0.00786629381522737,0.000405772215875058)
--(axis cs:0.012741787081786,0.00319293398368465)
--(axis cs:0.0154698955384146,0.00558826031322431)
--(axis cs:0.0106015778058684,0.00280119746317773)
--cycle;
\path [fill=yellow24724417]
(axis cs:0.0562166240135154,0.0347240054753193)
--(axis cs:0.0613364629315117,0.0379624181441895)
--(axis cs:0.0638626349914298,0.0402133987005247)
--(axis cs:0.0587515517371486,0.0369772911880742)
--cycle;
\path [fill=darkkhaki174189103]
(axis cs:0.0154698955384146,0.00558826031322431)
--(axis cs:0.0203853009309898,0.00840228032904644)
--(axis cs:0.0230832506938078,0.0107773961423499)
--(axis cs:0.0181753357794511,0.00796368355807538)
--cycle;
\path [fill=lightseagreen13173186]
(axis cs:-0.0176915244006146,-0.021696239064358)
--(axis cs:-0.0129716052230294,-0.0148712045037683)
--(axis cs:-0.0101298318771176,-0.0123949472527764)
--(axis cs:-0.014850565252673,-0.0192094321847436)
--cycle;
\path [fill=darkkhaki20618790]
(axis cs:0.0230832506938078,0.0107773961423499)
--(axis cs:0.0280386668623916,0.0136183416090215)
--(axis cs:0.0307063780667683,0.0159731929151536)
--(axis cs:0.0257587685154256,0.0131327641979293)
--cycle;
\path [fill=sandybrown23618576]
(axis cs:0.0307063780667683,0.0159731929151536)
--(axis cs:0.0357076622645547,0.0191235303707333)
--(axis cs:0.0383454543429563,0.021457257378883)
--(axis cs:0.0333518945707975,0.0183084524382559)
--cycle;
\path [fill=goldenrod25419753]
(axis cs:0.0383454543429563,0.021457257378883)
--(axis cs:0.043389034398484,0.0246376040403497)
--(axis cs:0.0459968074840673,0.0269501356275847)
--(axis cs:0.0409612830936154,0.0237715528285102)
--cycle;
\path [fill=lightseagreen56185157]
(axis cs:-0.0101298318771176,-0.0123949472527764)
--(axis cs:-0.00532737708950468,-0.00966171951612681)
--(axis cs:-0.00251576038447259,-0.00720566692334806)
--(axis cs:-0.0073117242070184,-0.00993931173835424)
--cycle;
\path [fill=dodgerblue16142210]
(axis cs:-0.0357818568095406,-0.0334676914393035)
--(axis cs:-0.0310813705289325,-0.0319335901060498)
--(axis cs:-0.0281624536466546,-0.0293847489080931)
--(axis cs:-0.0328555175852581,-0.0309230771403951)
--cycle;
\path [fill=dodgerblue19129213]
(axis cs:-0.0434012624872858,-0.0375506352605158)
--(axis cs:-0.0387327826325423,-0.0360336851497796)
--(axis cs:-0.0357818568095406,-0.0334676914393035)
--(axis cs:-0.0404431809981765,-0.0349890113532683)
--cycle;
\path [fill=lightseagreen6156207]
(axis cs:-0.0281624536466546,-0.0293847489080931)
--(axis cs:-0.0234450966812861,-0.0267325744654847)
--(axis cs:-0.0205562682282454,-0.0242038656397163)
--(axis cs:-0.0252678566661961,-0.0268571442064765)
--cycle;
\path [fill=mediumaquamarine98190134]
(axis cs:-0.00251576038447259,-0.00720566692334806)
--(axis cs:0.00232672637327339,-0.00444550456379214)
--(axis cs:0.00510810629729068,-0.00200971079790599)
--(axis cs:0.000272426794225699,-0.0047700809001401)
--cycle;
\path [fill=gold24621836]
(axis cs:0.0459968074840673,0.0269501356275847)
--(axis cs:0.0510828132629968,0.0301605844719186)
--(axis cs:0.0536604671228237,0.0324518492526164)
--(axis cs:0.0485828498169992,0.0292433967298998)
--cycle;
\path [fill=darkseagreen138190117]
(axis cs:0.00510810629729068,-0.00200971079790599)
--(axis cs:0.00999072431393291,0.000777453372551555)
--(axis cs:0.012741787081786,0.00319293398368465)
--(axis cs:0.00786629381522737,0.000405772215875058)
--cycle;
\path [fill=yellow24724417]
(axis cs:0.0536604671228237,0.0324518492526164)
--(axis cs:0.0587890289004111,0.0356924918143338)
--(axis cs:0.0613364629315117,0.0379624181441895)
--(axis cs:0.0562166240135154,0.0347240054753193)
--cycle;
\path [fill=darkkhaki174189103]
(axis cs:0.012741787081786,0.00319293398368465)
--(axis cs:0.0176646359302504,0.00600716736029786)
--(axis cs:0.0203853009309898,0.00840228032904644)
--(axis cs:0.0154698955384146,0.00558826031322431)
--cycle;
\path [fill=lightseagreen13173186]
(axis cs:-0.0205562682282454,-0.0242038656397163)
--(axis cs:-0.0158373436082757,-0.0173683443499566)
--(axis cs:-0.0129716052230294,-0.0148712045037683)
--(axis cs:-0.0176915244006146,-0.021696239064358)
--cycle;
\path [fill=darkkhaki20618790]
(axis cs:0.0203853009309898,0.00840228032904644)
--(axis cs:0.0253484804756933,0.0112436509318365)
--(axis cs:0.0280386668623916,0.0136183416090215)
--(axis cs:0.0230832506938078,0.0107773961423499)
--cycle;
\path [fill=sandybrown23618576]
(axis cs:0.0280386668623916,0.0136183416090215)
--(axis cs:0.033047629075637,0.0167701260420215)
--(axis cs:0.0357076622645547,0.0191235303707333)
--(axis cs:0.0307063780667683,0.0159731929151536)
--cycle;
\path [fill=goldenrod25419753]
(axis cs:0.0357076622645547,0.0191235303707333)
--(axis cs:0.0407592554962721,0.0223055580458458)
--(axis cs:0.043389034398484,0.0246376040403497)
--(axis cs:0.0383454543429563,0.021457257378883)
--cycle;
\path [fill=lightseagreen56185157]
(axis cs:-0.0129716052230294,-0.0148712045037683)
--(axis cs:-0.00816271988929747,-0.0121384977421148)
--(axis cs:-0.00532737708950468,-0.00966171951612681)
--(axis cs:-0.0101298318771176,-0.0123949472527764)
--cycle;
\path [fill=dodgerblue16142210]
(axis cs:-0.0387327826325423,-0.0360336851497796)
--(axis cs:-0.0340249151034265,-0.0345039365674527)
--(axis cs:-0.0310813705289325,-0.0319335901060498)
--(axis cs:-0.0357818568095406,-0.0334676914393035)
--cycle;
\path [fill=lightseagreen6156207]
(axis cs:-0.0310813705289325,-0.0319335901060498)
--(axis cs:-0.0263583147701858,-0.0292826325296925)
--(axis cs:-0.0234450966812861,-0.0267325744654847)
--(axis cs:-0.0281624536466546,-0.0293847489080931)
--cycle;
\path [fill=mediumaquamarine98190134]
(axis cs:-0.00532737708950468,-0.00966171951612681)
--(axis cs:-0.000478139714233852,-0.00690186633986621)
--(axis cs:0.00232672637327339,-0.00444550456379214)
--(axis cs:-0.00251576038447259,-0.00720566692334806)
--cycle;
\path [fill=gold24621836]
(axis cs:0.043389034398484,0.0246376040403497)
--(axis cs:0.0484833901085717,0.0278499690643617)
--(axis cs:0.0510828132629968,0.0301605844719186)
--(axis cs:0.0459968074840673,0.0269501356275847)
--cycle;
\path [fill=darkseagreen138190117]
(axis cs:0.00232672637327339,-0.00444550456379214)
--(axis cs:0.00721641630446464,-0.0016584369620813)
--(axis cs:0.00999072431393291,0.000777453372551555)
--(axis cs:0.00510810629729068,-0.00200971079790599)
--cycle;
\path [fill=yellow24724417]
(axis cs:0.0510828132629968,0.0301605844719186)
--(axis cs:0.0562200633310219,0.0334033795094589)
--(axis cs:0.0587890289004111,0.0356924918143338)
--(axis cs:0.0536604671228237,0.0324518492526164)
--cycle;
\path [fill=darkkhaki174189103]
(axis cs:0.00999072431393291,0.000777453372551555)
--(axis cs:0.0149209676042993,0.00359180362104869)
--(axis cs:0.0176646359302504,0.00600716736029786)
--(axis cs:0.012741787081786,0.00319293398368465)
--cycle;
\path [fill=lightseagreen13173186]
(axis cs:-0.0234450966812861,-0.0267325744654847)
--(axis cs:-0.0187273514669008,-0.0198866320683066)
--(axis cs:-0.0158373436082757,-0.0173683443499566)
--(axis cs:-0.0205562682282454,-0.0242038656397163)
--cycle;
\path [fill=darkkhaki20618790]
(axis cs:0.0176646359302504,0.00600716736029786)
--(axis cs:0.0226355336786451,0.00884886910612213)
--(axis cs:0.0253484804756933,0.0112436509318365)
--(axis cs:0.0203853009309898,0.00840228032904644)
--cycle;
\path [fill=sandybrown23618576]
(axis cs:0.0253484804756933,0.0112436509318365)
--(axis cs:0.0303650722892271,0.0143967944687713)
--(axis cs:0.033047629075637,0.0167701260420215)
--(axis cs:0.0280386668623916,0.0136183416090215)
--cycle;
\path [fill=goldenrod25419753]
(axis cs:0.033047629075637,0.0167701260420215)
--(axis cs:0.0381071910515033,0.0199537495875806)
--(axis cs:0.0407592554962721,0.0223055580458458)
--(axis cs:0.0357076622645547,0.0191235303707333)
--cycle;
\path [fill=lightseagreen56185157]
(axis cs:-0.0158373436082757,-0.0173683443499566)
--(axis cs:-0.0110220903787275,-0.0146362650557428)
--(axis cs:-0.00816271988929747,-0.0121384977421148)
--(axis cs:-0.0129716052230294,-0.0148712045037683)
--cycle;
\path [fill=lightseagreen6156207]
(axis cs:-0.0340249151034265,-0.0345039365674527)
--(axis cs:-0.0292962326774774,-0.0318543113477032)
--(axis cs:-0.0263583147701858,-0.0292826325296925)
--(axis cs:-0.0310813705289325,-0.0319335901060498)
--cycle;
\path [fill=mediumaquamarine98190134]
(axis cs:-0.00816271988929747,-0.0121384977421148)
--(axis cs:-0.00330679070469063,-0.00937905774724175)
--(axis cs:-0.000478139714233852,-0.00690186633986621)
--(axis cs:-0.00532737708950468,-0.00966171951612681)
--cycle;
\path [fill=gold24621836]
(axis cs:0.0407592554962721,0.0223055580458458)
--(axis cs:0.0458619207147894,0.0255197568550103)
--(axis cs:0.0484833901085717,0.0278499690643617)
--(axis cs:0.043389034398484,0.0246376040403497)
--cycle;
\path [fill=darkseagreen138190117]
(axis cs:-0.000478139714233852,-0.00690186633986621)
--(axis cs:0.00441856718567004,-0.00411499679717303)
--(axis cs:0.00721641630446464,-0.0016584369620813)
--(axis cs:0.00232672637327339,-0.00444550456379214)
--cycle;
\path [fill=yellow24724417]
(axis cs:0.0484833901085717,0.0278499690643617)
--(axis cs:0.0536292920800033,0.0310948369503491)
--(axis cs:0.0562200633310219,0.0334033795094589)
--(axis cs:0.0510828132629968,0.0301605844719186)
--cycle;
\path [fill=darkkhaki174189103]
(axis cs:0.00721641630446464,-0.0016584369620813)
--(axis cs:0.0121540029735861,0.00115593118938644)
--(axis cs:0.0149209676042993,0.00359180362104869)
--(axis cs:0.00999072431393291,0.000777453372551555)
--cycle;
\path [fill=lightseagreen13173186]
(axis cs:-0.0263583147701858,-0.0292826325296925)
--(axis cs:-0.0216419384112811,-0.0224263374480657)
--(axis cs:-0.0187273514669008,-0.0198866320683066)
--(axis cs:-0.0234450966812861,-0.0267325744654847)
--cycle;
\path [fill=darkkhaki20618790]
(axis cs:0.0149209676042993,0.00359180362104869)
--(axis cs:0.0198995363963533,0.00643374007595997)
--(axis cs:0.0226355336786451,0.00884886910612213)
--(axis cs:0.0176646359302504,0.00600716736029786)
--cycle;
\path [fill=sandybrown23618576]
(axis cs:0.0226355336786451,0.00884886910612213)
--(axis cs:0.0276597046141423,0.0120032814765926)
--(axis cs:0.0303650722892271,0.0143967944687713)
--(axis cs:0.0253484804756933,0.0112436509318365)
--cycle;
\path [fill=goldenrod25419753]
(axis cs:0.0303650722892271,0.0143967944687713)
--(axis cs:0.0354325565771113,0.0175819263869604)
--(axis cs:0.0381071910515033,0.0199537495875806)
--(axis cs:0.033047629075637,0.0167701260420215)
--cycle;
\path [fill=lightseagreen56185157]
(axis cs:-0.0187273514669008,-0.0198866320683066)
--(axis cs:-0.0139057952860643,-0.0171552893956527)
--(axis cs:-0.0110220903787275,-0.0146362650557428)
--(axis cs:-0.0158373436082757,-0.0173683443499566)
--cycle;
\path [fill=mediumaquamarine98190134]
(axis cs:-0.0110220903787275,-0.0146362650557428)
--(axis cs:-0.00615953042567394,-0.0118773448629519)
--(axis cs:-0.00330679070469063,-0.00937905774724175)
--(axis cs:-0.00816271988929747,-0.0121384977421148)
--cycle;
\path [fill=gold24621836]
(axis cs:0.0381071910515033,0.0199537495875806)
--(axis cs:0.0432181234192349,0.0231696974754259)
--(axis cs:0.0458619207147894,0.0255197568550103)
--(axis cs:0.0407592554962721,0.0223055580458458)
--cycle;
\path [fill=darkseagreen138190117]
(axis cs:-0.00330679070469063,-0.00937905774724175)
--(axis cs:0.00159687604733293,-0.00659249033708898)
--(axis cs:0.00441856718567004,-0.00411499679717303)
--(axis cs:-0.000478139714233852,-0.00690186633986621)
--cycle;
\path [fill=yellow24724417]
(axis cs:0.0458619207147894,0.0255197568550103)
--(axis cs:0.0510164363302571,0.0287666156931724)
--(axis cs:0.0536292920800033,0.0310948369503491)
--(axis cs:0.0484833901085717,0.0278499690643617)
--cycle;
\path [fill=darkkhaki174189103]
(axis cs:0.00441856718567004,-0.00411499679717303)
--(axis cs:0.00936344406200864,-0.00130071225527131)
--(axis cs:0.0121540029735861,0.00115593118938644)
--(axis cs:0.00721641630446464,-0.0016584369620813)
--cycle;
\path [fill=lightseagreen13173186]
(axis cs:-0.0292962326774774,-0.0318543113477032)
--(axis cs:-0.0245814193426978,-0.0249877348871137)
--(axis cs:-0.0216419384112811,-0.0224263374480657)
--(axis cs:-0.0263583147701858,-0.0292826325296925)
--cycle;
\path [fill=darkkhaki20618790]
(axis cs:0.0121540029735861,0.00115593118938644)
--(axis cs:0.0171401936036655,0.00399800341572235)
--(axis cs:0.0198995363963533,0.00643374007595997)
--(axis cs:0.0149209676042993,0.00359180362104869)
--cycle;
\path [fill=sandybrown23618576]
(axis cs:0.0198995363963533,0.00643374007595997)
--(axis cs:0.0249312338524262,0.00958932854993804)
--(axis cs:0.0276597046141423,0.0120032814765926)
--(axis cs:0.0226355336786451,0.00884886910612213)
--cycle;
\path [fill=goldenrod25419753]
(axis cs:0.0276597046141423,0.0120032814765926)
--(axis cs:0.0327350627231614,0.015189831853077)
--(axis cs:0.0354325565771113,0.0175819263869604)
--(axis cs:0.0303650722892271,0.0143967944687713)
--cycle;
\path [fill=lightseagreen56185157]
(axis cs:-0.0216419384112811,-0.0224263374480657)
--(axis cs:-0.0168141465826544,-0.0196958432805109)
--(axis cs:-0.0139057952860643,-0.0171552893956527)
--(axis cs:-0.0187273514669008,-0.0198866320683066)
--cycle;
\path [fill=mediumaquamarine98190134]
(axis cs:-0.0139057952860643,-0.0171552893956527)
--(axis cs:-0.00903666790166738,-0.014396998315221)
--(axis cs:-0.00615953042567394,-0.0118773448629519)
--(axis cs:-0.0110220903787275,-0.0146362650557428)
--cycle;
\path [fill=gold24621836]
(axis cs:0.0354325565771113,0.0175819263869604)
--(axis cs:0.0405517117409531,0.0207995362739919)
--(axis cs:0.0432181234192349,0.0231696974754259)
--(axis cs:0.0381071910515033,0.0199537495875806)
--cycle;
\path [fill=darkseagreen138190117]
(axis cs:-0.00615953042567394,-0.0118773448629519)
--(axis cs:-0.00124896317115214,-0.00909118630832533)
--(axis cs:0.00159687604733293,-0.00659249033708898)
--(axis cs:-0.00330679070469063,-0.00937905774724175)
--cycle;
\path [fill=yellow24724417]
(axis cs:0.0432181234192349,0.0231696974754259)
--(axis cs:0.0483812124909002,0.0264184630403497)
--(axis cs:0.0510164363302571,0.0287666156931724)
--(axis cs:0.0458619207147894,0.0255197568550103)
--cycle;
\path [fill=darkkhaki174189103]
(axis cs:0.00159687604733293,-0.00659249033708898)
--(axis cs:0.00654898778994006,-0.00377839352634848)
--(axis cs:0.00936344406200864,-0.00130071225527131)
--(axis cs:0.00441856718567004,-0.00411499679717303)
--cycle;
\path [fill=darkkhaki20618790]
(axis cs:0.00936344406200864,-0.00130071225527131)
--(axis cs:0.0143572052191213,0.00154139423645934)
--(axis cs:0.0171401936036655,0.00399800341572235)
--(axis cs:0.0121540029735861,0.00115593118938644)
--cycle;
\path [fill=sandybrown23618576]
(axis cs:0.0171401936036655,0.00399800341572235)
--(axis cs:0.0221793627941435,0.00715467273902549)
--(axis cs:0.0249312338524262,0.00958932854993804)
--(axis cs:0.0198995363963533,0.00643374007595997)
--cycle;
\path [fill=goldenrod25419753]
(axis cs:0.0249312338524262,0.00958932854993804)
--(axis cs:0.0300144151725002,0.0127772049901724)
--(axis cs:0.0327350627231614,0.015189831853077)
--(axis cs:0.0276597046141423,0.0120032814765926)
--cycle;
\path [fill=lightseagreen56185157]
(axis cs:-0.0245814193426978,-0.0249877348871137)
--(axis cs:-0.0197474615954318,-0.0222582039072901)
--(axis cs:-0.0168141465826544,-0.0196958432805109)
--(axis cs:-0.0216419384112811,-0.0224263374480657)
--cycle;
\path [fill=mediumaquamarine98190134]
(axis cs:-0.0168141465826544,-0.0196958432805109)
--(axis cs:-0.0119385174656529,-0.0169382933811919)
--(axis cs:-0.00903666790166738,-0.014396998315221)
--(axis cs:-0.0139057952860643,-0.0171552893956527)
--cycle;
\path [fill=gold24621836]
(axis cs:0.0327350627231614,0.015189831853077)
--(axis cs:0.0378623942769649,0.0184090142239267)
--(axis cs:0.0405517117409531,0.0207995362739919)
--(axis cs:0.0354325565771113,0.0175819263869604)
--cycle;
\path [fill=darkseagreen138190117]
(axis cs:-0.00903666790166738,-0.014396998315221)
--(axis cs:-0.00411926179144642,-0.011611358056676)
--(axis cs:-0.00124896317115214,-0.00909118630832533)
--(axis cs:-0.00615953042567394,-0.0118773448629519)
--cycle;
\path [fill=yellow24724417]
(axis cs:0.0405517117409531,0.0207995362739919)
--(axis cs:0.0457233320946566,0.0240501219491246)
--(axis cs:0.0483812124909002,0.0264184630403497)
--(axis cs:0.0432181234192349,0.0231696974754259)
--cycle;
\path [fill=darkkhaki174189103]
(axis cs:-0.00124896317115214,-0.00909118630832533)
--(axis cs:0.00371032586449786,-0.00627738402671206)
--(axis cs:0.00654898778994006,-0.00377839352634848)
--(axis cs:0.00159687604733293,-0.00659249033708898)
--cycle;
\path [fill=darkkhaki20618790]
(axis cs:0.00654898778994006,-0.00377839352634848)
--(axis cs:0.0115502659961628,-0.00093635691013139)
--(axis cs:0.0143572052191213,0.00154139423645934)
--(axis cs:0.00936344406200864,-0.00130071225527131)
--cycle;
\path [fill=sandybrown23618576]
(axis cs:0.0143572052191213,0.00154139423645934)
--(axis cs:0.0194037891094566,0.00469904656435422)
--(axis cs:0.0221793627941435,0.00715467273902549)
--(axis cs:0.0171401936036655,0.00399800341572235)
--cycle;
\path [fill=goldenrod25419753]
(axis cs:0.0221793627941435,0.00715467273902549)
--(axis cs:0.0272703145337075,0.0103437803027094)
--(axis cs:0.0300144151725002,0.0127772049901724)
--(axis cs:0.0249312338524262,0.00958932854993804)
--cycle;
\path [fill=mediumaquamarine98190134]
(axis cs:-0.0197474615954318,-0.0222582039072901)
--(axis cs:-0.0148653988735915,-0.0195015100871815)
--(axis cs:-0.0119385174656529,-0.0169382933811919)
--(axis cs:-0.0168141465826544,-0.0196958432805109)
--cycle;
\path [fill=gold24621836]
(axis cs:0.0300144151725002,0.0127772049901724)
--(axis cs:0.0351498745961042,0.0159978678289165)
--(axis cs:0.0378623942769649,0.0184090142239267)
--(axis cs:0.0327350627231614,0.015189831853077)
--cycle;
\path [fill=darkseagreen138190117]
(axis cs:-0.0119385174656529,-0.0169382933811919)
--(axis cs:-0.0070143365097987,-0.0141532836469157)
--(axis cs:-0.00411926179144642,-0.011611358056676)
--(axis cs:-0.00903666790166738,-0.014396998315221)
--cycle;
\path [fill=yellow24724417]
(axis cs:0.0378623942769649,0.0184090142239267)
--(axis cs:0.0430425016925927,0.0216613309377658)
--(axis cs:0.0457233320946566,0.0240501219491246)
--(axis cs:0.0405517117409531,0.0207995362739919)
--cycle;
\path [fill=darkkhaki174189103]
(axis cs:-0.00411926179144642,-0.011611358056676)
--(axis cs:0.000847144666968257,-0.00879795984777651)
--(axis cs:0.00371032586449786,-0.00627738402671206)
--(axis cs:-0.00124896317115214,-0.00909118630832533)
--cycle;
\path [fill=darkkhaki20618790]
(axis cs:0.00371032586449786,-0.00627738402671206)
--(axis cs:0.00871906541152587,-0.00343552413022491)
--(axis cs:0.0115502659961628,-0.00093635691013139)
--(axis cs:0.00654898778994006,-0.00377839352634848)
--cycle;
\path [fill=sandybrown23618576]
(axis cs:0.0115502659961628,-0.00093635691013139)
--(axis cs:0.0166042052379,0.00222217791874403)
--(axis cs:0.0194037891094566,0.00469904656435422)
--(axis cs:0.0143572052191213,0.00154139423645934)
--cycle;
\path [fill=goldenrod25419753]
(axis cs:0.0194037891094566,0.00469904656435422)
--(axis cs:0.0245024562312671,0.00788928769797691)
--(axis cs:0.0272703145337075,0.0103437803027094)
--(axis cs:0.0221793627941435,0.00715467273902549)
--cycle;
\path [fill=gold24621836]
(axis cs:0.0272703145337075,0.0103437803027094)
--(axis cs:0.0324138511300962,0.0135658290262948)
--(axis cs:0.0351498745961042,0.0159978678289165)
--(axis cs:0.0300144151725002,0.0127772049901724)
--cycle;
\path [fill=darkseagreen138190117]
(axis cs:-0.0148653988735915,-0.0195015100871815)
--(axis cs:-0.00993450951353073,-0.016717245965076)
--(axis cs:-0.0070143365097987,-0.0141532836469157)
--(axis cs:-0.0119385174656529,-0.0169382933811919)
--cycle;
\path [fill=yellow24724417]
(axis cs:0.0351498745961042,0.0159978678289165)
--(axis cs:0.0403384227461123,0.0192518239893277)
--(axis cs:0.0430425016925927,0.0216613309377658)
--(axis cs:0.0378623942769649,0.0184090142239267)
--cycle;
\path [fill=darkkhaki174189103]
(axis cs:-0.0070143365097987,-0.0141532836469157)
--(axis cs:-0.00204087486269848,-0.0113404018711877)
--(axis cs:0.000847144666968257,-0.00879795984777651)
--(axis cs:-0.00411926179144642,-0.011611358056676)
--cycle;
\path [fill=darkkhaki20618790]
(axis cs:0.000847144666968257,-0.00879795984777651)
--(axis cs:0.00586328755072377,-0.00595638628895479)
--(axis cs:0.00871906541152587,-0.00343552413022491)
--(axis cs:0.00371032586449786,-0.00627738402671206)
--cycle;
\path [fill=sandybrown23618576]
(axis cs:0.00871906541152587,-0.00343552413022491)
--(axis cs:0.0137802982747704,-0.000276210033179365)
--(axis cs:0.0166042052379,0.00222217791874403)
--(axis cs:0.0115502659961628,-0.00093635691013139)
--cycle;
\path [fill=goldenrod25419753]
(axis cs:0.0166042052379,0.00222217791874403)
--(axis cs:0.0217105303928724,0.00541345238615508)
--(axis cs:0.0245024562312671,0.00788928769797691)
--(axis cs:0.0194037891094566,0.00469904656435422)
--cycle;
\path [fill=gold24621836]
(axis cs:0.0245024562312671,0.00788928769797691)
--(axis cs:0.0296540170617916,0.0111126250876942)
--(axis cs:0.0324138511300962,0.0135658290262948)
--(axis cs:0.0272703145337075,0.0103437803027094)
--cycle;
\path [fill=yellow24724417]
(axis cs:0.0324138511300962,0.0135658290262948)
--(axis cs:0.0376107915161307,0.0168213304528977)
--(axis cs:0.0403384227461123,0.0192518239893277)
--(axis cs:0.0351498745961042,0.0159978678289165)
--cycle;
\path [fill=darkkhaki174189103]
(axis cs:-0.00993450951353073,-0.016717245965076)
--(axis cs:-0.00495405734441279,-0.0139049958731648)
--(axis cs:-0.00204087486269848,-0.0113404018711877)
--(axis cs:-0.0070143365097987,-0.0141532836469157)
--cycle;
\path [fill=darkkhaki20618790]
(axis cs:-0.00204087486269848,-0.0113404018711877)
--(axis cs:0.00298261099053641,-0.00849922711414316)
--(axis cs:0.00586328755072377,-0.00595638628895479)
--(axis cs:0.000847144666968257,-0.00879795984777651)
--cycle;
\path [fill=sandybrown23618576]
(axis cs:0.00586328755072377,-0.00595638628895479)
--(axis cs:0.0109317498545414,-0.00279639895812997)
--(axis cs:0.0137802982747704,-0.000276210033179365)
--(axis cs:0.00871906541152587,-0.00343552413022491)
--cycle;
\path [fill=goldenrod25419753]
(axis cs:0.0137802982747704,-0.000276210033179365)
--(axis cs:0.0188942217337803,0.00291599477776139)
--(axis cs:0.0217105303928724,0.00541345238615508)
--(axis cs:0.0166042052379,0.00222217791874403)
--cycle;
\path [fill=gold24621836]
(axis cs:0.0217105303928724,0.00541345238615508)
--(axis cs:0.0268700602104692,0.00863797851709313)
--(axis cs:0.0296540170617916,0.0111126250876942)
--(axis cs:0.0245024562312671,0.00788928769797691)
--cycle;
\path [fill=yellow24724417]
(axis cs:0.0296540170617916,0.0111126250876942)
--(axis cs:0.0348592989493405,0.0143695749422511)
--(axis cs:0.0376107915161307,0.0168213304528977)
--(axis cs:0.0324138511300962,0.0135658290262948)
--cycle;
\path [fill=darkkhaki20618790]
(axis cs:-0.00495405734441279,-0.0139049958731648)
--(axis cs:7.67086784113892e-05,-0.0110643353027557)
--(axis cs:0.00298261099053641,-0.00849922711414316)
--(axis cs:-0.00204087486269848,-0.0113404018711877)
--cycle;
\path [fill=sandybrown23618576]
(axis cs:0.00298261099053641,-0.00849922711414316)
--(axis cs:0.00805823603121424,-0.00533867546001927)
--(axis cs:0.0109317498545414,-0.00279639895812997)
--(axis cs:0.00586328755072377,-0.00595638628895479)
--cycle;
\path [fill=goldenrod25419753]
(axis cs:0.0109317498545414,-0.00279639895812997)
--(axis cs:0.0160532094381212,0.000396630378398626)
--(axis cs:0.0188942217337803,0.00291599477776139)
--(axis cs:0.0137802982747704,-0.000276210033179365)
--cycle;
\path [fill=gold24621836]
(axis cs:0.0188942217337803,0.00291599477776139)
--(axis cs:0.0240616629141195,0.00614160694617737)
--(axis cs:0.0268700602104692,0.00863797851709313)
--(axis cs:0.0217105303928724,0.00541345238615508)
--cycle;
\path [fill=yellow24724417]
(axis cs:0.0268700602104692,0.00863797851709313)
--(axis cs:0.0320836305614779,0.0118962772318349)
--(axis cs:0.0348592989493405,0.0143695749422511)
--(axis cs:0.0296540170617916,0.0111126250876942)
--cycle;
\path [fill=sandybrown23618576]
(axis cs:7.67086784113892e-05,-0.0110643353027557)
--(axis cs:0.00515942715550671,-0.00790333118860955)
--(axis cs:0.00805823603121424,-0.00533867546001927)
--(axis cs:0.00298261099053641,-0.00849922711414316)
--cycle;
\path [fill=goldenrod25419753]
(axis cs:0.00805823603121424,-0.00533867546001927)
--(axis cs:0.0131871670370723,-0.00214493031927934)
--(axis cs:0.0160532094381212,0.000396630378398626)
--(axis cs:0.0109317498545414,-0.00279639895812997)
--cycle;
\path [fill=gold24621836]
(axis cs:0.0160532094381212,0.000396630378398626)
--(axis cs:0.0212285019086123,0.00362322302693356)
--(axis cs:0.0240616629141195,0.00614160694617737)
--(axis cs:0.0188942217337803,0.00291599477776139)
--cycle;
\path [fill=yellow24724417]
(axis cs:0.0240616629141195,0.00614160694617737)
--(axis cs:0.0292834663174999,0.0094011521499971)
--(axis cs:0.0320836305614779,0.0118962772318349)
--(axis cs:0.0268700602104692,0.00863797851709313)
--cycle;
\path [fill=goldenrod25419753]
(axis cs:0.00515942715550671,-0.00790333118860955)
--(axis cs:0.010295762283796,-0.00470898194646978)
--(axis cs:0.0131871670370723,-0.00214493031927934)
--(axis cs:0.00805823603121424,-0.00533867546001927)
--cycle;
\path [fill=gold24621836]
(axis cs:0.0131871670370723,-0.00214493031927934)
--(axis cs:0.0183702482036558,0.00108253432138887)
--(axis cs:0.0212285019086123,0.00362322302693356)
--(axis cs:0.0160532094381212,0.000396630378398626)
--cycle;
\path [fill=yellow24724417]
(axis cs:0.0212285019086123,0.00362322302693356)
--(axis cs:0.0264584805085746,0.00688390946937788)
--(axis cs:0.0292834663174999,0.0094011521499971)
--(axis cs:0.0240616629141195,0.00614160694617737)
--cycle;
\path [fill=gold24621836]
(axis cs:0.010295762283796,-0.00470898194646978)
--(axis cs:0.0154865669554432,-0.00148075681159201)
--(axis cs:0.0183702482036558,0.00108253432138887)
--(axis cs:0.0131871670370723,-0.00214493031927934)
--cycle;
\path [fill=yellow24724417]
(axis cs:0.0183702482036558,0.00108253432138887)
--(axis cs:0.0236083416257854,0.00434425379437161)
--(axis cs:0.0264584805085746,0.00688390946937788)
--(axis cs:0.0212285019086123,0.00362322302693356)
--cycle;
\path [fill=yellow24724417]
(axis cs:0.0154865669554432,-0.00148075681159201)
--(axis cs:0.0207327122304472,0.00178188444557029)
--(axis cs:0.0236083416257854,0.00434425379437161)
--(axis cs:0.0183702482036558,0.00108253432138887)
--cycle;

\end{axis}

\begin{axis}[
colorbar,
colorbar style={ylabel={}},
colormap={mymap}{[1pt]
  rgb(0pt)=(0.2081,0.1663,0.5292);
  rgb(1pt)=(0.2116238095,0.1897809524,0.5776761905);
  rgb(2pt)=(0.212252381,0.2137714286,0.6269714286);
  rgb(3pt)=(0.2081,0.2386,0.6770857143);
  rgb(4pt)=(0.1959047619,0.2644571429,0.7279);
  rgb(5pt)=(0.1707285714,0.2919380952,0.779247619);
  rgb(6pt)=(0.1252714286,0.3242428571,0.8302714286);
  rgb(7pt)=(0.0591333333,0.3598333333,0.8683333333);
  rgb(8pt)=(0.0116952381,0.3875095238,0.8819571429);
  rgb(9pt)=(0.0059571429,0.4086142857,0.8828428571);
  rgb(10pt)=(0.0165142857,0.4266,0.8786333333);
  rgb(11pt)=(0.032852381,0.4430428571,0.8719571429);
  rgb(12pt)=(0.0498142857,0.4585714286,0.8640571429);
  rgb(13pt)=(0.0629333333,0.4736904762,0.8554380952);
  rgb(14pt)=(0.0722666667,0.4886666667,0.8467);
  rgb(15pt)=(0.0779428571,0.5039857143,0.8383714286);
  rgb(16pt)=(0.079347619,0.5200238095,0.8311809524);
  rgb(17pt)=(0.0749428571,0.5375428571,0.8262714286);
  rgb(18pt)=(0.0640571429,0.5569857143,0.8239571429);
  rgb(19pt)=(0.0487714286,0.5772238095,0.8228285714);
  rgb(20pt)=(0.0343428571,0.5965809524,0.819852381);
  rgb(21pt)=(0.0265,0.6137,0.8135);
  rgb(22pt)=(0.0238904762,0.6286619048,0.8037619048);
  rgb(23pt)=(0.0230904762,0.6417857143,0.7912666667);
  rgb(24pt)=(0.0227714286,0.6534857143,0.7767571429);
  rgb(25pt)=(0.0266619048,0.6641952381,0.7607190476);
  rgb(26pt)=(0.0383714286,0.6742714286,0.743552381);
  rgb(27pt)=(0.0589714286,0.6837571429,0.7253857143);
  rgb(28pt)=(0.0843,0.6928333333,0.7061666667);
  rgb(29pt)=(0.1132952381,0.7015,0.6858571429);
  rgb(30pt)=(0.1452714286,0.7097571429,0.6646285714);
  rgb(31pt)=(0.1801333333,0.7176571429,0.6424333333);
  rgb(32pt)=(0.2178285714,0.7250428571,0.6192619048);
  rgb(33pt)=(0.2586428571,0.7317142857,0.5954285714);
  rgb(34pt)=(0.3021714286,0.7376047619,0.5711857143);
  rgb(35pt)=(0.3481666667,0.7424333333,0.5472666667);
  rgb(36pt)=(0.3952571429,0.7459,0.5244428571);
  rgb(37pt)=(0.4420095238,0.7480809524,0.5033142857);
  rgb(38pt)=(0.4871238095,0.7490619048,0.4839761905);
  rgb(39pt)=(0.5300285714,0.7491142857,0.4661142857);
  rgb(40pt)=(0.5708571429,0.7485190476,0.4493904762);
  rgb(41pt)=(0.609852381,0.7473142857,0.4336857143);
  rgb(42pt)=(0.6473,0.7456,0.4188);
  rgb(43pt)=(0.6834190476,0.7434761905,0.4044333333);
  rgb(44pt)=(0.7184095238,0.7411333333,0.3904761905);
  rgb(45pt)=(0.7524857143,0.7384,0.3768142857);
  rgb(46pt)=(0.7858428571,0.7355666667,0.3632714286);
  rgb(47pt)=(0.8185047619,0.7327333333,0.3497904762);
  rgb(48pt)=(0.8506571429,0.7299,0.3360285714);
  rgb(49pt)=(0.8824333333,0.7274333333,0.3217);
  rgb(50pt)=(0.9139333333,0.7257857143,0.3062761905);
  rgb(51pt)=(0.9449571429,0.7261142857,0.2886428571);
  rgb(52pt)=(0.9738952381,0.7313952381,0.266647619);
  rgb(53pt)=(0.9937714286,0.7454571429,0.240347619);
  rgb(54pt)=(0.9990428571,0.7653142857,0.2164142857);
  rgb(55pt)=(0.9955333333,0.7860571429,0.196652381);
  rgb(56pt)=(0.988,0.8066,0.1793666667);
  rgb(57pt)=(0.9788571429,0.8271428571,0.1633142857);
  rgb(58pt)=(0.9697,0.8481380952,0.147452381);
  rgb(59pt)=(0.9625857143,0.8705142857,0.1309);
  rgb(60pt)=(0.9588714286,0.8949,0.1132428571);
  rgb(61pt)=(0.9598238095,0.9218333333,0.0948380952);
  rgb(62pt)=(0.9661,0.9514428571,0.0755333333);
  rgb(63pt)=(0.9763,0.9831,0.0538)
},
point meta max=20.0000000616968,
point meta min=3.00000005641589,
tick align=outside,
tick pos=left,
x grid style={darkgray176},
xmin=90, xmax=110,
xtick style={color=black},
y grid style={darkgray176},
ymin=90, ymax=110,
ytick style={color=black}
]
\addplot graphics [includegraphics cmd=\pgfimage,xmin=90, xmax=110, ymin=90, ymax=110] {mytikz-000.png};
\end{axis}

\end{tikzpicture}

\end{document}